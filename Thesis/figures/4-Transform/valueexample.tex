\begin{figure}%
    \centering

    \subfloat[AUIPC]{
        \begin{tabular}[h]{>{\ttfamily\color{UniRed}}r >{\ttfamily}l >{\ttfamily\color{UniGrey}}l >{\slshape\color{UniRed}}l >{\slshape\color{UniRed}}l >{\slshape\color{UniRed}}l >{\slshape} l}
            \hline
            \hline
            (n + 0) & add & D & imm & pc &  & rdAUIPC \\
            \hline
            \hline
        \end{tabular}
    }

    %    \caption[Examples for instruction execution]{Instruction execution for chosen instructions}\label{fig:valueexample}
    %\end{figure}
    %\begin{figure}%
    %    \ContinuedFloat
    %    \centering

    \subfloat[JALR]{
        \begin{tabular}[h]{>{\ttfamily\color{UniRed}}r >{\ttfamily}l >{\ttfamily\color{UniGrey}}l >{\slshape\color{UniRed}}l >{\slshape\color{UniRed}}l >{\slshape\color{UniRed}}l >{\slshape} l}
            \hline
            \hline
            (n + 0) & add   & AS & pc                         & pcInc                   &  & nextPc      \\
            (n + 0) & add   & D  & imm                        & rs1val                  &  & pcJALR64pre \\
            (n + 0) & and   & D  & \textcolor{Black}{-}1Const & pcJALR64pre             &  & pcJALR64    \\
            (n + 0) & slice & AS & pcJALR64                   & \textcolor{UniBlue}{15} &  & pcJALR      \\
            (n + 0) & uext  & D  & nextPc                     & \textcolor{UniBlue}{48} &  & rdJALR      \\
            \hline
            \hline
            \multicolumn{7}{l}{\todo{pc overflow erwähnen}}
        \end{tabular}
    }

    \caption[]{Instruction execution for chosen instructions}\label{fig:valueexample}
\end{figure}
\begin{figure}%
    \ContinuedFloat
    \centering

    \subfloat[BEQ]{
        \begin{tabular}[h]{>{\ttfamily\color{UniRed}}r >{\ttfamily}l >{\ttfamily\color{UniGrey}}l >{\slshape\color{UniRed}}l >{\slshape\color{UniRed}}l >{\slshape\color{UniRed}}l >{\slshape} l}
            \hline
            \hline
            (n + 0) & add   & AS   & pc        & pcInc                   &                        & nextPc    \\
            (n + 0) & slice & AS   & imm       & \textcolor{UniBlue}{15} & \textcolor{UniBlue}{0} & branchImm \\
            (n + 0) & add   & AS   & pc        & branchImm               &                        & branchPc  \\
            (n + 0) & eq    & Bool & rs1val    & rs2val                  &                        & isBEQcond \\
            (n + 0) & ite   & AS   & isBEQcond & branchPc                & nextPc                 & pcBEQ     \\
            \hline
            \hline
        \end{tabular}
    }

    \caption[]{Instruction execution for chosen instructions}\label{fig:valueexample}
\end{figure}
\begin{figure}%
    \ContinuedFloat
    \centering

    \subfloat[LHU]{
        \begin{tabular}[h]{>{\ttfamily\color{UniRed}}r >{\ttfamily}l >{\ttfamily\color{UniGrey}}l >{\slshape\color{UniRed}}l >{\slshape\color{UniRed}}l >{\slshape\color{UniRed}}l >{\slshape} l}
            \hline
            \hline
            (n + 0) & add    & D  & rs1val     & imm                     &                        & 1stAddrPre \\
            (n + 0) & slice  & AS & 1stAddrPre & \textcolor{UniBlue}{15} & \textcolor{UniBlue}{0} & 1stAddr    \\
            (n + 0) & add    & AS & 1stAddr    & addressInc              &                        & 2ndAddr    \\
            (n + 0) & read   & B  & memory     & 1stAddr                 &                        & loadB1     \\
            (n + 0) & read   & B  & memory     & 2ndAddr                 &                        & loadB2     \\
            (n + 0) & concat & H  & loadB2     & loadB1                  &                        & loadB2B1   \\
            (n + 0) & uext   & D  & loadB2B1   & \textcolor{UniBlue}{48} & \textcolor{UniBlue}{0} & rdLHU      \\
            \hline
            \hline
        \end{tabular}
    }

    \caption[]{Instruction execution for chosen instructions}\label{fig:valueexample}
\end{figure}
\begin{figure}%
    \ContinuedFloat
    \centering

    \subfloat[SD]{
        \begin{tabular}[h]{>{\ttfamily\color{UniRed}}r >{\ttfamily}l >{\ttfamily\color{UniGrey}}l >{\slshape\color{UniRed}}l >{\slshape\color{UniRed}}l >{\slshape\color{UniRed}}l >{\slshape} l}
            \hline
            \hline
            (n + 0) & add   & D   & rs1val     & imm                     &                         & 1stAddrPre \\
            (n + 0) & slice & AS  & 1stAddrPre & \textcolor{UniBlue}{15} & \textcolor{UniBlue}{0}  & 1stAddr    \\
            (n + 0) & add   & AS  & 1stAddr    & addressInc              &                         & 2ndAddr    \\
            (n + 0) & add   & AS  & 2ndAddr    & addressInc              &                         & 3rdAddr    \\
            (n + 0) & add   & AS  & 3rdAddr    & addressInc              &                         & 4thAddr    \\
            (n + 0) & add   & AS  & 4thAddr    & addressInc              &                         & 5thAddr    \\
            (n + 0) & add   & AS  & 5thAddr    & addressInc              &                         & 6thAddr    \\
            (n + 0) & add   & AS  & 6thAddr    & addressInc              &                         & 7thAddr    \\
            (n + 0) & add   & AS  & 7thAddr    & addressInc              &                         & 8thAddr    \\
            \\
            (n + 0) & slice & B   & rs2val     & \textcolor{UniBlue}{7}  & \textcolor{UniBlue}{0}  & storeB1    \\
            (n + 0) & slice & B   & rs2val     & \textcolor{UniBlue}{15} & \textcolor{UniBlue}{8}  & storeB2    \\
            (n + 0) & slice & B   & rs2val     & \textcolor{UniBlue}{23} & \textcolor{UniBlue}{16} & storeB3    \\
            (n + 0) & slice & B   & rs2val     & \textcolor{UniBlue}{31} & \textcolor{UniBlue}{24} & storeB4    \\
            (n + 0) & slice & B   & rs2val     & \textcolor{UniBlue}{39} & \textcolor{UniBlue}{32} & storeB5    \\
            (n + 0) & slice & B   & rs2val     & \textcolor{UniBlue}{47} & \textcolor{UniBlue}{40} & storeB6    \\
            (n + 0) & slice & B   & rs2val     & \textcolor{UniBlue}{55} & \textcolor{UniBlue}{48} & storeB7    \\
            (n + 0) & slice & B   & rs2val     & \textcolor{UniBlue}{63} & \textcolor{UniBlue}{56} & storeB8    \\
            \\
            (n + 0) & write & Mem & memory     & 1stAddr                 & storeB1                 & memorySB   \\
            (n + 0) & write & Mem & memorySB   & 2ndAddr                 & storeB2                 & memorySH   \\
            (n + 0) & write & Mem & memorySH   & 3rdAddr                 & storeB3                 & memoryB3   \\
            (n + 0) & write & Mem & memoryB3   & 4thAddr                 & storeB4                 & memorySW   \\
            (n + 0) & write & Mem & memorySW   & 5thAddr                 & storeB5                 & memoryB5   \\
            (n + 0) & write & Mem & memoryB5   & 6thAddr                 & storeB6                 & memoryB6   \\
            (n + 0) & write & Mem & memoryB6   & 7thAddr                 & storeB7                 & memoryB7   \\
            (n + 0) & write & Mem & memoryB7   & 8thAddr                 & storeB8                 & memorySD   \\
            \hline
            \hline
        \end{tabular}
    }

    \caption[]{Instruction execution for chosen instructions}\label{fig:valueexample}
\end{figure}
\begin{figure}%
    \ContinuedFloat
    \centering

    \subfloat[ANDI]{
        \begin{tabular}[h]{>{\ttfamily\color{UniRed}}r >{\ttfamily}l >{\ttfamily\color{UniGrey}}l >{\slshape\color{UniRed}}l >{\slshape\color{UniRed}}l >{\slshape\color{UniRed}}l >{\slshape} l}
            \hline
            \hline
            (n + 0) & and & D & rs1val & imm &  & rdANDI \\
            \hline
            \hline
        \end{tabular}
    }

    \caption[]{Instruction execution for chosen instructions}\label{fig:valueexample}
\end{figure}
\begin{figure}%
    \ContinuedFloat
    \centering

    \subfloat[SLLIW]{
        \begin{tabular}[h]{>{\ttfamily\color{UniRed}}r >{\ttfamily}l >{\ttfamily\color{UniGrey}}l >{\slshape\color{UniRed}}l >{\slshape\color{UniRed}}l >{\slshape\color{UniRed}}l >{\slshape} l}
            \hline
            \hline
            (n + 0) & and   & W & imm32    & 5Bitmask                &                        & shamtIW    \\
            (n + 0) & slice & W & rs1val   & \textcolor{UniBlue}{31} & \textcolor{UniBlue}{0} & rs1val32   \\
            (n + 0) & sll   & W & rs1val32 & shamtIW                 &                        & rdSLLIWpre \\
            (n + 0) & sext  & D & rs1val32 & 32                      &                        & rdSLLIW    \\
            \hline
            \hline
        \end{tabular}
    }

    \caption[]{Instruction execution for chosen instructions}\label{fig:valueexample}
\end{figure}
\begin{figure}%
    \ContinuedFloat
    \centering

    \subfloat[SLT]{
        \begin{tabular}[h]{>{\ttfamily\color{UniRed}}r >{\ttfamily}l >{\ttfamily\color{UniGrey}}l >{\slshape\color{UniRed}}l >{\slshape\color{UniRed}}l >{\slshape\color{UniRed}}l >{\slshape} l}
            \hline
            \hline
            (n + 0) & slt  & Bool & rs1val   & rs2val                  &  & rdSLTpre \\
            (n + 0) & uext & D    & rdSLTpre & \textcolor{UniBlue}{63} &  & rdSLT    \\
            \hline
            \hline
        \end{tabular}
    }

    \caption[]{Instruction execution for chosen instructions}\label{fig:valueexample}
\end{figure}
\begin{figure}%
    \ContinuedFloat
    \centering

    \subfloat[SUBW]{
        \begin{tabular}[h]{>{\ttfamily\color{UniRed}}r >{\ttfamily}l >{\ttfamily\color{UniGrey}}l >{\slshape\color{UniRed}}l >{\slshape\color{UniRed}}l >{\slshape\color{UniRed}}l >{\slshape} l}
            \hline
            \hline
            (n + 0) & slice & W & rs1val    & \textcolor{UniBlue}{31} & \textcolor{UniBlue}{0} & rs1val32  \\
            (n + 0) & slice & W & rs2val    & \textcolor{UniBlue}{31} & \textcolor{UniBlue}{0} & rs2val32  \\
            (n + 0) & sub   & W & rs1val32  & rs2val32                &                        & rdSUBWpre \\
            (n + 0) & sext  & D & rdSUBWpre & \textcolor{UniBlue}{32} &                        & rdSUBW    \\
            \hline
            \hline
        \end{tabular}
    }

    \caption[]{Instruction execution for chosen instructions}\label{fig:valueexample}
\end{figure}