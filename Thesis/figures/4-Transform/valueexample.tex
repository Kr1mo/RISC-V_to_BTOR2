\begin{figure}%
    \centering

    \subfloat[AUIPC]{
        \begin{tabular}[h]{>{\ttfamily\color{UniRed}}r >{\ttfamily}l >{\ttfamily\color{UniGrey}}l >{\slshape\color{UniRed}}l >{\slshape\color{UniRed}}l >{\slshape\color{UniRed}}l >{\slshape} l}
            \hline
            \hline
            n & add & D & imm & pc &  & rdAUIPC \\
            \hline
            \hline
        \end{tabular}
        \label{auipc}
    }

    \subfloat[JALR]{
        \begin{tabular}[h]{>{\ttfamily\color{UniRed}}r >{\ttfamily}l >{\ttfamily\color{UniGrey}}l >{\slshape\color{UniRed}}l >{\slshape\color{UniRed}}l >{\slshape\color{UniRed}}l >{\slshape} l}
            \hline
            \hline
            (n + 0) & add   & AS & pc                                          & pcInc                    &  & nextPc \\
            (n + 1) & add   & D  & imm                                         & rs1val                   &  &        \\
            (n + 2) & and   & D  & \textcolor{Black}{\upshape\ttfamily-}1Const & \upshape\ttfamily(n + 1) &  &        \\
            (n + 3) & slice & AS & \upshape\ttfamily(n + 2)                    & \textcolor{UniBlue}{15}  &  & pcJALR \\
            (n + 4) & uext  & D  & nextPc                                      & \textcolor{UniBlue}{48}  &  & rdJALR \\
            \hline
            \hline
        \end{tabular}
        \label{jalr}
    }

    \subfloat[BEQ]{
        \begin{tabular}[h]{>{\ttfamily\color{UniRed}}r >{\ttfamily}l >{\ttfamily\color{UniGrey}}l >{\slshape\color{UniRed}}l >{\slshape\color{UniRed}}l >{\slshape\color{UniRed}}l >{\slshape} l}
            \hline
            \hline
            (n + 0) & add   & AS   & pc                       & pcInc                   &                        & nextPc   \\
            (n + 1) & slice & AS   & imm                      & \textcolor{UniBlue}{15} & \textcolor{UniBlue}{0} & ImmAS    \\
            (n + 2) & add   & AS   & pc                       & ImmAS                   &                        & pcBranch \\
            (n + 3) & eq    & Bool & rs1val                   & rs2val                  &                        &          \\
            (n + 4) & ite   & AS   & \upshape\ttfamily(n + 3) & pcBranch                & nextPc                 & pcBEQ    \\
            \hline
            \hline
        \end{tabular}
        \label{beq}
    }

    \subfloat[LHU]{
        \begin{tabular}[h]{>{\ttfamily\color{UniRed}}r >{\ttfamily}l >{\ttfamily\color{UniGrey}}l >{\slshape\color{UniRed}}l >{\slshape\color{UniRed}}l >{\slshape\color{UniRed}}l >{\slshape} l}
            \hline
            \hline
            (n + 0) & add    & D  & rs1val                   & imm                      &                        &       \\
            (n + 1) & slice  & AS & \upshape\ttfamily(n + 0) & \textcolor{UniBlue}{15}  & \textcolor{UniBlue}{0} &       \\
            (n + 2) & add    & AS & \upshape\ttfamily(n + 1) & addressInc               &                        &       \\
            (n + 3) & read   & B  & memory                   & \upshape\ttfamily(n + 1) &                        &       \\
            (n + 4) & read   & B  & memory                   & \upshape\ttfamily(n + 2) &                        &       \\
            (n + 5) & concat & H  & \upshape\ttfamily(n + 3) & \upshape\ttfamily(n + 4) &                        &       \\
            (n + 6) & uext   & D  & \upshape\ttfamily(n + 5) & \textcolor{UniBlue}{48}  & \textcolor{UniBlue}{0} & rdLHU \\
            \hline
            \hline
        \end{tabular}
        \label{lhu}
    }

    \subfloat[ANDI]{
        \begin{tabular}[h]{>{\ttfamily\color{UniRed}}r >{\ttfamily}l >{\ttfamily\color{UniGrey}}l >{\slshape\color{UniRed}}l >{\slshape\color{UniRed}}l >{\slshape\color{UniRed}}l >{\slshape} l}
            \hline
            \hline
            n & and & D & rs1val & imm &  & rdANDI \\
            \hline
            \hline
        \end{tabular}
        \label{andi}
    }

    \subfloat[SLT]{
        \begin{tabular}[h]{>{\ttfamily\color{UniRed}}r >{\ttfamily}l >{\ttfamily\color{UniGrey}}l >{\slshape\color{UniRed}}l >{\slshape\color{UniRed}}l >{\slshape\color{UniRed}}l >{\slshape} l}
            \hline
            \hline
            (n + 0) & slt  & Bool & rs1val                   & rs2val                  &  &       \\
            (n + 1) & uext & D    & \upshape\ttfamily(n + 0) & \textcolor{UniBlue}{63} &  & rdSLT \\
            \hline
            \hline
        \end{tabular}
        \label{slt}
    }

    \caption[Examples for Instruction Execution]{Instruction Execution
        for chosen Instructions}\label{fig:valueexample}
\end{figure}
\begin{figure}%
    \ContinuedFloat
    \centering

    \subfloat[SD]{
        \begin{tabular}[h]{>{\ttfamily\color{UniRed}}r >{\ttfamily}l >{\ttfamily\color{UniGrey}}l >{\slshape\color{UniRed}}l >{\slshape\color{UniRed}}l >{\slshape\color{UniRed}}l >{\slshape} l}
            \hline
            \hline
            (n + 0)  & add   & D   & rs1val                    & imm                      &                           &          \\
            (n + 1)  & slice & AS  & \upshape\ttfamily(n + 0)  & \textcolor{UniBlue}{15}  & \textcolor{UniBlue}{0}    &          \\
            (n + 2)  & add   & AS  & \upshape\ttfamily(n + 1)  & addressInc               &                           &          \\
            (n + 3)  & add   & AS  & \upshape\ttfamily(n + 2)  & addressInc               &                           &          \\
            (n + 4)  & add   & AS  & \upshape\ttfamily(n + 3)  & addressInc               &                           &          \\
            (n + 5)  & add   & AS  & \upshape\ttfamily(n + 4)  & addressInc               &                           &          \\
            (n + 6)  & add   & AS  & \upshape\ttfamily(n + 5)  & addressInc               &                           &          \\
            (n + 7)  & add   & AS  & \upshape\ttfamily(n + 6)  & addressInc               &                           &          \\
            (n + 8)  & add   & AS  & \upshape\ttfamily(n + 7)  & addressInc               &                           &          \\
            \\
            (n + 9)  & slice & B   & rs2val                    & \textcolor{UniBlue}{7}   & \textcolor{UniBlue}{0}    &          \\
            (n + 10) & slice & B   & rs2val                    & \textcolor{UniBlue}{15}  & \textcolor{UniBlue}{8}    &          \\
            (n + 11) & slice & B   & rs2val                    & \textcolor{UniBlue}{23}  & \textcolor{UniBlue}{16}   &          \\
            (n + 12) & slice & B   & rs2val                    & \textcolor{UniBlue}{31}  & \textcolor{UniBlue}{24}   &          \\
            (n + 13) & slice & B   & rs2val                    & \textcolor{UniBlue}{39}  & \textcolor{UniBlue}{32}   &          \\
            (n + 14) & slice & B   & rs2val                    & \textcolor{UniBlue}{47}  & \textcolor{UniBlue}{40}   &          \\
            (n + 15) & slice & B   & rs2val                    & \textcolor{UniBlue}{55}  & \textcolor{UniBlue}{48}   &          \\
            (n + 16) & slice & B   & rs2val                    & \textcolor{UniBlue}{63}  & \textcolor{UniBlue}{56}   &          \\
            \\
            (n + 17) & write & Mem & memory                    & \upshape\ttfamily(n + 1) & \upshape\ttfamily(n + 9)  & memorySB \\
            (n + 18) & write & Mem & memorySB                  & \upshape\ttfamily(n + 2) & \upshape\ttfamily(n + 10) & memorySH \\
            (n + 19) & write & Mem & memorySH                  & \upshape\ttfamily(n + 3) & \upshape\ttfamily(n + 11) &          \\
            (n + 20) & write & Mem & \upshape\ttfamily(n + 19) & \upshape\ttfamily(n + 4) & \upshape\ttfamily(n + 12) & memorySW \\
            (n + 21) & write & Mem & memorySW                  & \upshape\ttfamily(n + 5) & \upshape\ttfamily(n + 13) &          \\
            (n + 22) & write & Mem & \upshape\ttfamily(n + 21) & \upshape\ttfamily(n + 6) & \upshape\ttfamily(n + 14) &          \\
            (n + 23) & write & Mem & \upshape\ttfamily(n + 22) & \upshape\ttfamily(n + 7) & \upshape\ttfamily(n + 15) &          \\
            (n + 24) & write & Mem & \upshape\ttfamily(n + 23) & \upshape\ttfamily(n + 8) & \upshape\ttfamily(n + 16) & memorySD \\
            \hline
            \hline
        \end{tabular}
        \label{sd}
    }

    \subfloat[SLLIW]{
        \begin{tabular}[h]{>{\ttfamily\color{UniRed}}r >{\ttfamily}l >{\ttfamily\color{UniGrey}}l >{\slshape\color{UniRed}}l >{\slshape\color{UniRed}}l >{\slshape\color{UniRed}}l >{\slshape} l}
            \hline
            \hline
            (n + 0) & and   & W & imm32                    & 5Bitmask                 &                        &         \\
            (n + 1) & slice & W & rs1val                   & \textcolor{UniBlue}{31}  & \textcolor{UniBlue}{0} &         \\
            (n + 2) & sll   & W & \upshape\ttfamily(n + 1) & \upshape\ttfamily(n + 0) &                        &         \\
            (n + 3) & sext  & D & \upshape\ttfamily(n + 2) & 32                       &                        & rdSLLIW \\
            \hline
            \hline
        \end{tabular}
        \label{slliw}
    }

    \subfloat[SUBW]{
        \begin{tabular}[h]{>{\ttfamily\color{UniRed}}r >{\ttfamily}l >{\ttfamily\color{UniGrey}}l >{\slshape\color{UniRed}}l >{\slshape\color{UniRed}}l >{\slshape\color{UniRed}}l >{\slshape} l}
            \hline
            \hline
            (n + 0) & slice & W & rs1val                   & \textcolor{UniBlue}{31}  & \textcolor{UniBlue}{0} &        \\
            (n + 1) & slice & W & rs2val                   & \textcolor{UniBlue}{31}  & \textcolor{UniBlue}{0} &        \\
            (n + 2) & sub   & W & \upshape\ttfamily(n + 0) & \upshape\ttfamily(n + 1) &                        &        \\
            (n + 3) & sext  & D & \upshape\ttfamily(n + 2) & \textcolor{UniBlue}{32}  &                        & rdSUBW \\
            \hline
            \hline
        \end{tabular}
        \label{subw}
    }

    \caption[]{Continuation of Instruction Execution for chosen
        Instructions}
\end{figure}