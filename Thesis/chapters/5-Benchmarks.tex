\chapter{Benchmarks}\label{chap:benchmarks}
With a model implemented, I can test how good it runs. I run my
benchmarks on an Intel Core i5-6200U. Each test was run five times
and the resulting times averaged. I devised two base tests formed
from four RISC-V instructions as shown in
\figref{fig:base_test_risc}. I chose to implement one test with and
one without memory operation to be able to measure the impact memory
operations have in the model. For this reason the program for these
tests are very similar. Both have three instructions forming a loop
and one instruction as a \enquote{workhorse}. My naming for these
also derive from this instruction as add and write\_mem. This program
is set into a state. An example for this is found in
\figref{fig:bench_example}, where the add program is set up for 256
loops. In this, x1 acts as a loop limiter, x2 as a loop counter and
x3 as an accumulator. The instructions are set in the first bytes of
the memory. In difference to this, the memory operation test uses x2
also as an address to store the first byte of a register and x3 as
this register. Both tests are also implemented with rising loop
counts up to 2048 to form a reference on how they behave with more
iterations to process.

\begin{figure}
    \centering
    \begin{tabular}{c | c >{\slshape}l}
        \multicolumn{2}{c}{\ttfamily bge x2 x1 0x10} & jump out of program if x1 = x2                                                     \\
        \ttfamily add x3 x3 x2                       &                                & either (add counter onto x3)                      \\
                                                     & \ttfamily sb x3 0x14(x2)       & or (store the first byte of x3 at counter + 0x14) \\
        \multicolumn{2}{c}{\ttfamily addi x2 x2 0x1} & increment counter in x2                                                            \\
        \multicolumn{2}{c}{\ttfamily jalr x0 x0 0x0} & jump back to address 0                                                             \\
    \end{tabular}
    \caption[Base Test Cases for Benchmarks]{Base Test Cases for the Benchmarks}\label{fig:base_test_risc}
\end{figure}
\begin{figure}
    \centering
    \begin{verbatim}
REGISTERS:
PC:0
x1:100
x2:0

MEMORY:
0:001158E3 # BGE x2 x1 0x10
4:002181B3 # ADD x3 x3 x2
8:00110113 # ADDI x2 x2 1
c:00000067 # JALR x0 x0 0
    \end{verbatim}
    \caption[Example benchmark state]{Example state for benchmark
        add\_0256}\label{fig:bench_example}
\end{figure}

Further on I tested the impact of initialized memory on the run time
of the test. For this I added tests with the prefix fullmem, where I
filled the memory addresses 0x18 to 0xfff with the pattern '0101'.
The first four words of the memory are not filled for there lies the
program of the test. The fifth could also be filled instead of being
left zero-initialized, but I decided against it because the test is
designed to terminate by jumping to this address and not finding a
valid instruction there. Leaving it zero-initialized guarantees this.
To further extend on this test, I also added some tests on double the
initialized memory, so filled to the address 0x1fff.\todo{benchmark
    them, add times}

I also tested on the impact of the address space size. For this I
generated models from add\_0256, add\_1024, write\_mem\_0256 and
write\_mem\_1024 with extended address space. These tests have the
prefix \enquote{extaddr\_\textcolor{Green}{x}}, where
\textcolor{Green}{x} is the maximum length of an address in this
model in bits.

Finally, I also implemented the base add tests similar to the BTOR2
model F.Schrögendorfer describes in his master thesis \cite[Chapter
    8]{bmcOfLockless}. These tests have the prefix \enquote{nopc}, as the
pc is abstracted to activation flags for each instruction. I show the
generation of the model on the example of nopc\_add\_0256. First of,
we need the sorts necessary for RISC-V:\\ \centering
\begin{tabular}[h]{>{\ttfamily\color{UniRed}}r >{\ttfamily}l >{\ttfamily}l >{\slshape\color{UniBlue}}l >{\slshape\color{UniBlue}}l >{\slshape\color{UniRed}}l >{\slshape} l}
    1 & sort & bitvec & 1  &   &  & bool         \\
    2 & sort & bitvec & 8  &   &  & memcell      \\
    3 & sort & bitvec & 16 &   &  & addressspace \\
    4 & sort & bitvec & 64 &   &  & register     \\
    5 & sort & array  & 3  & 2 &  & Memory       \\
\end{tabular}\\
Next we need the relevant constants:\\
\begin{figure}[h!]
    \centering
    \begin{tabular}[h]{>{\ttfamily\color{UniRed}}r >{\ttfamily}l >{\ttfamily\color{UniGrey}}l >{\slshape\color{UniRed}}l >{\slshape\color{UniRed}}l >{\slshape\color{UniRed}}l >{\slshape} l}
        6  & zero   & 1 &                          &  &  & false                             \\
        7  & one    & 1 &                          &  &  & true                              \\
        8  & zero   & 2 &                          &  &  & ; for initializing empty memory   \\
        9  & zero   & 4 &                          &  &  & ; for initializing empty register \\
        10 & zero   & 3 &                          &  &  & pcinit                            \\
        11 & consth & 3 & \textcolor{UniBlue}{4}   &  &  & pcinc                             \\
        12 & consth & 3 & \textcolor{UniBlue}{10}  &  &  & instr1pcmod                       \\
        13 & consth & 4 & \textcolor{UniBlue}{100} &  &  &                                   \\
    \end{tabular}
\end{figure}
\section{Results}
\begin{table}
    \centering
    \begin{tabular}{r|rr|rr|r}
        loops & \multicolumn{2}{c|}{base} & \multicolumn{2}{c|}{fullmem} & nopc                      \\
              & add                       & writemem                     & add    & writemem & add   \\ \hline
        0256  & 2.635                     & 2.877                        & 9.759  & 13.344   & 0.136 \\
        0512  & 6.195                     & 7.306                        & 16.402 & 24.209   & 0.268 \\
        0768  & 10.802                    & 13.283                       & 24.093 & 36.388   & 0.414 \\
        1024  & 16.306                    & 21.004                       & 32.732 & 50.376   & 0.569 \\
        1280  & 23.032                    & 30.200                       & 42.410 & 65.746   & 0.728 \\
        1536  & 30.669                    & 41.262                       & 52.961 & 83.036   & 0.898 \\
        1792  & 39.463                    & 53.940                       & 64.598 & 101.475  & 1.075 \\
        2048  & 48.944                    & 68.521                       & 77.084 & 122.189  & 1.276 \\
    \end{tabular}
    \caption{Times of iterations based benchmarks}\label{tab:time_iter}
\end{table}

\begin{table}
    \centering
    \begin{tabular}{r|rrrrr}
        bits of address space & 16     & 17            & 18     & 19     & 20     \\\hline
        add\_0256             & 2.635  & 2.632         & 2.626  & 2.626  & 2.624  \\
        add\_1024             & 16.306 & 16.464        & 16.511 & 16.452 & 16.460 \\

        writemem\_0256        & 2.877  & 2.88          & 2.890  & 2.889  & 2.890  \\
        writemem\_1024        & 21.004 & 21.131 21.215 & 21.181 & 21.163          \\

    \end{tabular}
    \caption{Times of extended address space benchmarks}\label{tab:time_extaddr}
\end{table}

\begin{figure}
    \centering
    \begin{tikzpicture}
        \begin{axis}[
                xlabel={Iterations},
                ylabel={Time [s]},
                xmin=0206, xmax=2098,
                ymin=0, ymax=120,
                xtick={0256, 0512,1024,2048},
                ytick={0,20,40,60,80,100,120},
                legend pos=north west,
                ymajorgrids=true,
                grid style=dashed,
            ]

            \addplot[ color=UniRed, mark=*, dashed] coordinates { (0256, 2.635)
                    (0512, 6.195) (0768, 10.802) (1024, 16.306) (1280, 23.032) (1536,
                    30.669) (1792, 39.463) (2048, 48.944) };

            \addplot[ color= UniBlue, mark=triangle*, dashed] coordinates {
                    (0256,2.877) (0512,7.306) (0768,13.283) (1024,21.004) (1280,30.200)
                    (1536,41.262) (1792,53.940) (2048,68.521) };

            \addplot[ color= UniRed, mark=square*, dashed] coordinates {
                    (0256,9.759) (0512,16.402) (0768,24.093) (1024,32.732) (1280,42.410)
                    (1536,52.961) (1792,64.598) (2048,77.084) };

            \addplot[ color= UniBlue, mark=diamond*, dashed] coordinates { (0256,
                    13.344) (0512, 24.209) (0768, 36.388) (1024, 50.376) (1280, 65.746)
                    (1536, 83.036) (1792, 101.475) (2048, 122.189) }; \legend{add,
                writemem, fullmem\_add, fullmem\_writemem}

        \end{axis}
    \end{tikzpicture}

    \caption[Plotted iterations based benchmark
        times]{\tabref{tab:time_iter} plotted}
\end{figure}