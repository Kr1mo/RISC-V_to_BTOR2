\chapter{Benchmarks}\label{chap:benchmarks}
With a model implemented, I can test how good it runs. I run my
benchmarks on an Intel Core i5-6200U. Each test was run five times
and the resulting times averaged. I devised two base tests formed
from four RISC-V instructions as shown in
\figref{fig:base_test_risc}. Both have three instructions forming a
loop and one instruction as a \enquote{workhorse}. This program is
set into a state like shown in \figref{fig:bench_example}. In this,
x1 acts as a loop limiter, x2 as a loop counter and x3 as an
accumulator.

\begin{figure}
    \centering
    \begin{tabular}{c | c >{\slshape}l}
        \multicolumn{2}{c}{\ttfamily bge x2 x1 0x10} & jump out of program if x1 = x2                                                     \\
        \ttfamily add x3 x3 x2                       &                                & either (add counter onto x3)                      \\
                                                     & \ttfamily sb x3 0x14(x2)       & or (store the first byte of x3 at counter + 0x14) \\
        \multicolumn{2}{c}{\ttfamily addi x2 x2 0x1} & increment counter in x2                                                            \\
        \multicolumn{2}{c}{\ttfamily jalr x0 x0 0x0} & jump back to address 0                                                             \\
    \end{tabular}
    \caption[Base Test Cases for Benchmarks]{Base Test Cases for the Benchmarks}\label{fig:base_test_risc}
\end{figure}
\input{figures/5-Benchmarks/add_0256_state.tex}
\begin{figure}
        \centering
        \begin{tikzpicture}
                \begin{axis}[
                                xlabel={Iterations},
                                ylabel={Time [s]},
                                xmin=0206, xmax=2098,
                                ymin=0, ymax=120,
                                xtick={0256, 0512,1024,2048},
                                ytick={0,20,40,60,80,100,120},
                                legend pos=north west,
                                ymajorgrids=true,
                                grid style=dashed,
                        ]

                        \addplot[ color=UniRed, mark=*, dashed] coordinates { (0256, 2.635)
                                        (0512, 6.195) (0768, 10.802) (1024, 16.306) (1280, 23.032) (1536,
                                        30.669) (1792, 39.463) (2048, 48.944) };

                        \addplot[ color= UniBlue, mark=triangle*, dashed] coordinates {
                                        (0256,2.877) (0512,7.306) (0768,13.283) (1024,21.004) (1280,30.200)
                                        (1536,41.262) (1792,53.940) (2048,68.521) };

                        \addplot[ color= UniRed, mark=square*, dashed] coordinates {
                                        (0256,9.759) (0512,16.402) (0768,24.093) (1024,32.732) (1280,42.410)
                                        (1536,52.961) (1792,64.598) (2048,77.084) };

                        \addplot[ color= UniBlue, mark=diamond*, dashed] coordinates { (0256,
                                        13.344) (0512, 24.209) (0768, 36.388) (1024, 50.376) (1280, 65.746)
                                        (1536, 83.036) (1792, 101.475) (2048, 122.189) }; \legend{add,
                                writemem, fullmem\_add, fullmem\_writemem}

                \end{axis}
        \end{tikzpicture}

        \caption[Plotted Iterations-based Benchmark
                Times]{\tabref{tab:time_iter} plotted}\label{fig:benchIterPlot}
\end{figure}
\section{Results}
\begin{table}
    \centering
    \begin{tabular}{r|rr|rr|r}
        \multirow{2}{*}{loops} & \multicolumn{2}{c|}{base} & \multicolumn{2}{c|}{fullmem} & nopc                      \\
                               & add                       & writemem                     & add    & writemem & add   \\ \hline
        0256                   & 2.635                     & 2.877                        & 9.759  & 13.344   & 0.136 \\
        0512                   & 6.195                     & 7.306                        & 16.402 & 24.209   & 0.268 \\
        0768                   & 10.802                    & 13.283                       & 24.093 & 36.388   & 0.414 \\
        1024                   & 16.306                    & 21.004                       & 32.732 & 50.376   & 0.569 \\
        1280                   & 23.032                    & 30.200                       & 42.410 & 65.746   & 0.728 \\
        1536                   & 30.669                    & 41.262                       & 52.961 & 83.036   & 0.898 \\
        1792                   & 39.463                    & 53.940                       & 64.598 & 101.475  & 1.075 \\
        2048                   & 48.944                    & 68.521                       & 77.084 & 122.189  & 1.276 \\
    \end{tabular}
    \caption{Times of iterations based benchmarks}\label{tab:time_iter}
\end{table}

\begin{table}
    \centering
    %    \begin{tabular}{r|rrrrr}
    %        bits of address space & 16     & 17     & 18     & 19     & 20     \\\hline
    %        add\_0256             & 2.635  & 2.632  & 2.626  & 2.626  & 2.624  \\
    %        add\_1024             & 16.306 & 16.464 & 16.511 & 16.452 & 16.460 \\
    %
    %        writemem\_0256        & 2.877  & 2.88   & 2.890  & 2.889  & 2.890  \\
    %        writemem\_1024        & 21.004 & 21.131 & 21.215 & 21.181 & 21.163 \\
    %
    %    \end{tabular}
    \begin{tabular}{l|l|l|l|l}
        pc width & add\_0256 & add\_1024 & writemem\_0256 & writemem\_1024 \\ \hline
        16       & 2.635     & 16.306    & 2.877          & 21.004         \\
        17       & 2.632     & 16.464    & 2.88           & 21.131         \\
        18       & 2.626     & 16.511    & 2.890          & 21.215         \\
        19       & 2.626     & 16.452    & 2.889          & 21.181         \\
        20       & 2.624     & 16.460    & 2.890          & 21.163         \\
    \end{tabular}
    \caption{Times of extended address space benchmarks}\label{tab:time_extaddr}
\end{table}
