\chapter*{Abstract}

As RISC-V continues to grow in attention \cite{eu}, I explore the
feasibility of using bounded model checking to run
RISC-V\cite{riscv-isa}. For this, I implemented a tool to transform a
RISC-V processor state into the model checking format
BTOR2\cite{btor2}. To test the correctness of the model I compared
the execution of one instruction of random-generated processor states
between the model and a RISC-V-simulator \cite{repoSim}. I further
implemented tests to benchmark the runtime of model checkers in
relation to the number of executed RISC-V instructions and the
available address space. Furthermore, I tested for runtime
differences between executing memory and non-memory instructions and
between zero-initializatised and already filled memory. I tried to
use the currently best BTOR2 model checkers \cite{HWMCC} and came to
the result that the better the model checker, the worse it performed
with my model.