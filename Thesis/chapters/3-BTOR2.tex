\chapter{BTOR2}\label{chap:btor2}

The second foundation of my benchmarks is BTOR2, a word-level model checking
format published by A. Niemetz et al. \cite{btor2}.

\section{Model Checking}
\todo{Write something about model checking...}

\section{The BTOR2 Language}
Generally in BTOR2, every line represents either a sort or a node, where
normally the line number acts as an identifier. A sort behaves similar to a
type as with it, either the length of a bitvector or the size of an array of
bitvectors is defined. Nodes on the other hand represent a value of a defined
sort and come as constants, operations or constraints. These values can later
on be referenced by the node identifier, so the line number. The syntax of
BTOR2 can be found at \cite[figure 1]{btor2} and corresponding operators in
\cite[table 1]{btor2}

Key features of BTOR2 include its ability to operate sequentially, which makes
the implementation of a RISC-V structure highly convenient. The main feature is
the \texttt{state} operator, which defines a node that is sequentially updated.
With an \texttt{init} node, this state can be assigned an initial value, and
with a \texttt{next} node, the sequentially next state can be defined. Finally,
constraints can be used to specify endpoints for a model. These endpoints may
indicate that something unintended has occurred or that the intended
information has been found. In either case, the resulting model is provided as
a witness.

\section{The BTOR2 Witness}\label{witness}
After receiving a witness, it must be interpreted. On the second line of a
witness, the constraint that was triggered is specified. Subsequently, for each
sequential iteration, the witness first presents—marked with
\texttt{\#}\textit{x}, where \textit{x} is the iteration number—a
representation of all states in the current iteration. Second, marked with
\texttt{@}\textit{x}, all inputs for the iteration are listed.

\todo{Maybe a bit more, its a bit bare bones}