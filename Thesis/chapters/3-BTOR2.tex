\chapter{BTOR2}\label{chap:btor2}

The second foundation of my benchmarks is BTOR2, a word-level model checking format \cite{btor2}.

\section{Model Checking}
\todo{Schreib was über modelchecking...}

\section{The BTOR2 Language}
Generally in BTOR2, every line represents either a sort or a node, where normally the line number akts as an identifier.
A sort behaves similar to a type as with it, either the length of a bitvector or the size of an array of bitvectors is defined.
Nodes on the other hand represent a value of a defined sort and come as constants, operations or constraints.


\section{The BTOR2 Witness}\label{witness}