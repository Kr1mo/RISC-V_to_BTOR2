\chapter{BTOR2}\label{chap:btor2}

The second foundation of my benchmarks is BTOR2, a word-level model
checking format published by A. Niemetz et al. \cite{btor2}.

\section{Model Checking}
\todo{Write something about model checking...} \todo{related work :)}

\section{The BTOR2 Language}
In BTOR2, each line represents either a sort or a node, with the line
number typically serving as an identifier. A sort functions similarly
to a type, defining either the length of a bitvector or the size of
an array of bitvectors. Nodes represent values of a defined sort and
can be constants, operations, or constraints. These values can be
referenced by their node identifier, i.e., the line number. The
syntax of BTOR2 is detailed in \cite[Figure 1]{btor2}, and the
available operators are listed in \cite[Table 1]{btor2}.

Key features of BTOR2 include its support for sequential operations,
which facilitates the implementation of a RISC-V structure. The
primary feature is the \texttt{state} operator, which defines a node
that is updated sequentially. An \texttt{init} node assigns an
initial value to this state, while a \texttt{next} node specifies its
subsequent value. Constraints can be used to define endpoints for a
model, indicating either the occurrence of an unintended event or the
discovery of the intended information. In both cases, the resulting
model produces a witness.

\section{The BTOR2 Witness}\label{witness}
Upon receiving a witness, it must be interpreted. The second line of
a witness specifies the constraint that was triggered. For each
sequential iteration, the witness first presents—marked with
\texttt{\#}\textit{x}, where \textit{x} is the iteration number—a
representation of all states in the current iteration. Second, marked
with \texttt{@}\textit{x}, all inputs for the iteration are listed.

\todo{Expand this section for greater detail}