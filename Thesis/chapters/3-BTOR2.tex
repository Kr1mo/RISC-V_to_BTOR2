\chapter{BTOR2}\label{chap:btor2}

The second foundation of my benchmarks is BTOR2, a word-level model
checking format published by A. Niemetz et al. \cite{btor2}. Before
explaining the format, an overview of bounded model checking is
necessary.

\section{Bounded Model Checking}
Bounded model checking (BMC) is a formal verification technique
employed to detect errors in hardware or software systems by
systematically exploring the state space of a finite-state model up
to a specified bound, typically defined by the number of iterations
or steps. As described by A. Biere in the \enquote{Handbook of
    Satisfiability}, BMC is primarily utilized for falsification and
testing, with a focus on identifying violations of temporal
properties \cite{BiereBMC}. Nevertheless, BMC can also be extended to
prove properties within the given bound.

In practice, BMC translates the verification problem into a
satisfiability problem, determining whether a property violation can
occur within the specified bound. The model comprises a finite state
machine and a set of properties to be verified. The model checker
systematically explores all possible state transitions up to the
bound and evaluates whether the property holds. If a violation is
detected, the tool generates a \emph{witness}, which is a trace
demonstrating how the property is violated. If no violation is found
within the bound, the system is considered safe up to that bound,
although this does not guarantee correctness for all possible
executions.

For word-level hardware model checking, the BTOR2 format has become a
de facto standard for describing models and is currently used in the
\enquote{Hardware Model Checking Competition} \cite{HWMCC}. Another
format is AIGER \cite{AIGER}, from which BTOR2 is derived
\cite{btor2}, and which is used in the bit-level track of the HWMCC
\cite{HWMCC}. In software model checking, the \enquote{Competition on
    Software Verification} utilizes the C and Java programming languages
as input formats \cite{SVCOMP25}. Additionally, a translator from
BTOR2 models to C programs has been presented to bridge the gap
between hardware and software verification \cite{btor2c}. As this
work focuses on BTOR2, the following section provides a detailed
overview of the format.

\section{The BTOR2 Language}
In BTOR2, each line represents either a sort or a node, with the line
number typically serving as an identifier. A sort functions similarly
to a type, defining either the length of a bitvector or the size of
an array of bitvectors. Nodes represent values of a defined sort and
can be constants, operations, constraints, or properties. These
values can be referenced by their node identifier, i.e., the line
number. The syntax of BTOR2 is detailed in \cite[Figure 1]{btor2},
and the available operators are listed in \cite[Table 1]{btor2}.

Key features of BTOR2 include its support for sequential operations,
which facilitates the implementation of a RISC-V structure. The
primary feature is the \texttt{state} operator, which defines a node
that is updated sequentially. An \texttt{init} node assigns an
initial value to this state, while a \texttt{next} node specifies its
subsequent value. \texttt{bad} nodes can be used to define endpoints
for a model, indicating either the occurrence of an unintended event
or, as in this work, the discovery of the intended information. In
both cases, the resulting model produces a witness. Additionally, an
\texttt{input} node allows an input to the model, with assignments to
this node handled by the model checker. These inputs can be
constrained by \texttt{constraint} nodes, which describe invariants
for the inputs of the model. An example model with these sequential
nodes is shown in \figref{fig:btor2example}.
\begin{figure}
    \begin{center}
        \begin{tabular}[h!]{>{\ttfamily\color{UniRed}}r >{\ttfamily}l >{\ttfamily\color{UniGrey}}l >{\ttfamily\color{UniRed}}l >{\slshape}l | >{\itshape}l}
            \multicolumn{5}{c|}{BTOR2 model} & \multicolumn{1}{c}{Comments}                                                                                    \\\hline
            1                                & sort                         & \color{Black}bitvec & \color{UniBlue} 1  &      & bit \enquote{type}             \\
            2                                & sort                         & \color{Black}bitvec & \color{UniBlue} 8  &      & byte \enquote{type}            \\
            3                                & constd                       & 2                   & \color{UniBlue} 99 &      & comparison constant            \\
            4                                & zero                         & 2                   &                    &      &                                \\
            5                                & input                        & 2                   &                    & i1   & input i1                       \\
            6                                & input                        & 2                   &                    & i2   & input i2                       \\
            7                                & eq                           & 1                   & 6 5                &      &                                \\
            8                                & constraint                   & \color{UniRed}-6    &                    &      & i1 $\neq$ i2 must hold         \\
            9                                & state                        & 2                   &                    & accu & sequential node accu           \\
            10                               & init                         & 2                   & 9 4                &      & initialization accu            \\
            11                               & add                          & 2                   & 6 5                &      & i1 + i2                        \\
            12                               & next                         & 2                   & 9 11               &      & next accu is i1 + i2           \\
            13                               & eq                           & 1                   & 3 9                &      & accu $=$ constant              \\
            14                               & bad                          & \color{UniRed}13    &                    &      & property: accu $\neq$ constant \\

        \end{tabular}
        \caption[An example BTOR2 model]{An example BTOR2 model finding two numbers that are not equal and added together equal 99}\label{fig:btor2example}
    \end{center}
\end{figure}
This model can now be checked by a model checker, which should produce a witness with an assignment of the two inputs $i1$ and $i2$ such that $s1 = i1 + i2 = constant$.
Let us examine this witness in more detail.

\section{The BTOR2 Witness}\label{witness}

Running the model in \figref{fig:bench_example} through BtorMC
\cite{btor2} with the option \texttt{--trace-gen-full} produces the
complete witness shown in \figref{fig:witnessexample}. The syntax of
BTOR2 witnesses is described in \cite[Figure 2]{btor2}, but I will
explain the example witness in \figref{fig:witnessexample} line by
line for clarity.

\begin{figure}
    \begin{verbatim}
sat
b0
#0
0 00000000 accu#0
@0
0 11101000 i1@0
1 01111011 i2@0
#1
0 01100011 accu0#1
@1
0 00000100 i1@1
1 00000000 i2@1
.
\end{verbatim}
    \caption[An example BTOR2 witness]{The complete witness of the model
        in \figref{fig:btor2example}}\label{fig:witnessexample}
\end{figure}

The witness begins with \texttt{sat}, indicating that a property in
the model is satisfiable. The second line specifies the property that
was violated. In this case, only one \texttt{bad} node exists in the
model, and the witness shows that \texttt{b0}, meaning the first
occurring \texttt{bad} node, was violated.

Subsequent lines list the iterations of the counterexample for the
property. For each sequential iteration, the witness first
presents—marked with \texttt{\#}\textit{x}, where \textit{x} is the
iteration number—a representation of all states in the current
iteration with their respective values in binary. Second, marked with
\texttt{@}\textit{x}, all inputs for the iteration are listed,
similar to the states. Note that the nodes of the inputs and the
state in \figref{fig:btor2example} include a symbol, which is used in
the witness to name the states and inputs. Examining the witness, it
is evident that the counterexample requires two iterations to reach
the violated property. In the first iteration, the state
\texttt{accu} is initialized with 0, and \texttt{i1} and \texttt{i2}
are assigned values that add up to 99 (01100011b). In the next
iteration, this sum is set as the new value for \texttt{accu}, which
violates the property.