\chapter{Introduction}\label{chap:intro}

RISC-V, an open-source instruction set architecture, has gained
significant attention due to its flexibility and extensibility and
was even mentioned by the European Union for broader adoption
\cite{eu}. Ensuring the correctness of RISC-V implementations is
therefore of great importance, particularly in safety-critical
applications.

Model checking has emerged as a powerful method for formally
verifying hardware and software systems. Among the available model
checking formats, BTOR2 has become a widely adopted standard for
word-level hardware verification. As RISC-V is a processor
architecture, it is a fitting choice. This thesis investigates the
feasibility and effectiveness of applying model checking to RISC-V
using the BTOR2 format.

The primary objective of this work is to develop tools that translate
RISC-V processor states into BTOR2 models, enabling systematic
verification, and to evaluate the performance of the model. This is
extended by a set of tools to check if the model functions correctly.
By implementing a suite of test cases and analyzing the results
across different model checkers, this thesis aims to provide insights
into the suitability of BTOR2-based model checking for RISC-V and to
identify potential limitations and advantages of this approach.

The following chapters present the theoretical foundations, the
methodology for transforming RISC-V states to BTOR2, the benchmarking
process, and a discussion of the results.