\chapter{Transforming RISC-V to BTOR2}\label{chap:riscv_to_btor2}

\todo{explain naming conventions for the model nodes}

This chapter addresses the main problem of the thesis: transforming a RISC-V
state into the BTOR2 format for benchmarking purposes. My primary reference for
this endeavor is F. Schrögendorfer's master's thesis, \dq Bounded Model
Checking in Lockless Programs\dq \cite{bmcOfLockless}, in which he describes,
among other topics, an encoding concept for a minimal machine in a
multiprocessor context \cite[Chapter 2]{bmcOfLockless} and two approaches to
next-state logic: a functional \cite[Chapter 6]{bmcOfLockless} and a relational
\cite[Chapter 7]{bmcOfLockless} approach. I will focus on the relational
approach; a discussion of both approaches can be found in
\secref{sec:funcVSrel}.

\section{The Concept}
To successfully execute a RISC-V instruction, three fundamental steps must
occur in sequence:
\begin{itemize}
    \item Fetch the current instruction from memory
    \item Identify the instruction
    \item Execute the instruction
\end{itemize}
Due to the fixed instruction length of RISC-V, as mentioned in
\secref{sec:riscvIsa}, fetching the current instruction is straightforward.
Ultimately, we want a node that retrieves a $word$ from memory at the location
specified by $pc$.

For basic identification, the $opcode$ must be extracted and checked. Depending
on the opcode, further distinctions between instructions require extracting and
checking $funct3$ and, if necessary, $funct7$. Ultimately, we want a node for
each instruction, which holds a boolean value indicating whether this
instruction was fetched.

To execute the instruction, we need to extract the values of the immediate
$imm$ and, if used, the registers $rs1$ and $rs2$. All instructions only modify
$rd$, $pc$, or memory. Therefore, the next-state logic can be generalized for
these three cases.

Memory is only modified when a store instruction is identified. As all store
instructions share the same type, computing the memory address is consistent
across them. The final step is overwriting the memory at this address.

For the $pc$, except for jump commands, it always increments to point to the
next instruction. The two unconditional jumps, \texttt{JAL} and \texttt{JALR},
must be handled separately. For branch instructions, after determining whether
the relevant condition for the instruction holds, we can generalize, as all
branch instructions execute the same operation from this point onward.

With $rd$, generalization across instructions is not feasible. However, we can
generalize across all possible registers by adding a check in each register's
update function to determine whether the register in question is $rd$.

\section{Encoding}
For better visualisation in the BTOR2 code I will mark all sort-ids in
\textcolor{UniGrey}{grey}, all node-ids in \textcolor{UniRed}{red} and all
non-id numbers \textcolor{UniBlue}{blue}. As described in the BTOR2 syntax
\cite[Figure 1]{btor2}, each line can get an accompanying symbol. Sadly those
cant be used as an alias to the line numbers, but for increased clarity, in the
following figures I will use them as such aliases. With this I can also start
each new figure with the relative line number \texttt{n} and it makes it
feasible to describe processes with algorithms. It is implied that \texttt{n}
is sufficiently incremented after adding to the model so that ids will not
overlap. In the following, I will describe how I construct a BTOR2 model for a
RISC-V state file.

\subsection{Constants}
First off, I added the sorts and non-progressive constants needed into the
BTOR2 model as seen in \figref{fig:constants}. This is extended by a set of
progressive constants used for comparison e.g. against the register number.
\algoref{alg:progressiveconsts} describes how they are added.

Of note is the Representation of the memory as an array of addressable memory
cells of each 1byte. Obviously, the set address space of 16bit is magnitudes
away of the expected address space of 64bit, but representing a 64bit
addressable memory with its resulting $2^{64}B \approx 18 Exabyte$ is not
implementable. Therefore, as I needed a feasible amount of memory space, I
artificially chose a 16bit address space as a soft minimum. With $~65kB$ and
therefore programs with possibly $>10000$ instructions I deemed this memory
sufficient for most use cases. Despite this, the encoding is implemented in
such a way that the address space can be altered with. \todo{benchmark
    auswirkungen von memory size}

\begin{figure}
    \centering
    \ttfamily
    \begin{tabular}{>{\color{UniRed}}r l l l l >{\slshape} l}
        \hline
        \hline
        \textcolor{UniGrey}{1} & sort   & bitvec                    & \textcolor{UniBlue}{1}      &                        & Bool          \\
        \textcolor{UniGrey}{2} & sort   & bitvec                    & \textcolor{UniBlue}{16}     &                        & AS            \\
        \textcolor{UniGrey}{3} & sort   & bitvec                    & \textcolor{UniBlue}{8}      &                        & B             \\
        \textcolor{UniGrey}{4} & sort   & bitvec                    & \textcolor{UniBlue}{16}     &                        & H             \\
        \textcolor{UniGrey}{5} & sort   & bitvec                    & \textcolor{UniBlue}{32}     &                        & W             \\
        \textcolor{UniGrey}{6} & sort   & bitvec                    & \textcolor{UniBlue}{64}     &                        & D             \\
        \textcolor{UniGrey}{7} & sort   & array                     & \textcolor{UniGrey}{2}      & \textcolor{UniGrey}{3} & Mem           \\
        \\
        8                      & one    & \textcolor{UniGrey}{Bool} &                             &                        & true          \\
        9                      & zero   & \textcolor{UniGrey}{Bool} &                             &                        & false         \\
        10                     & one    & \textcolor{UniGrey}{AS}   &                             &                        & addressInc    \\
        11                     & constd & \textcolor{UniGrey}{AS}   & \textcolor{UniBlue}{4}      &                        & pcInc         \\
        12                     & zero   & \textcolor{UniGrey}{B}    &                             &                        & emptyCell     \\
        13                     & one    & \textcolor{UniGrey}{W}    &                             &                        & bitPicker     \\
        14                     & zero   & \textcolor{UniGrey}{D}    &                             &                        & emptyReg      \\
        \\
        15                     & consth & \textcolor{UniGrey}{W}    & \textcolor{UniBlue}{01F}    &                        & 5Bitmask      \\
        16                     & consth & \textcolor{UniGrey}{W}    & \textcolor{UniBlue}{03F}    &                        & 6Bitmask      \\
        17                     & consth & \textcolor{UniGrey}{W}    & \textcolor{UniBlue}{07F}    &                        & 7Bitmask      \\
        18                     & consth & \textcolor{UniGrey}{W}    & \textcolor{UniBlue}{0FFF}   &                        & 12Bitmask     \\
        19                     & consth & \textcolor{UniGrey}{W}    & \textcolor{UniBlue}{0FFFFF} &                        & 20Bitmask     \\
        \\
        20                     & constd & \textcolor{UniGrey}{W}    & \textcolor{UniBlue}{7}      &                        & shiftToRd     \\
        21                     & constd & \textcolor{UniGrey}{W}    & \textcolor{UniBlue}{15}     &                        & shiftToRs1    \\
        22                     & constd & \textcolor{UniGrey}{W}    & \textcolor{UniBlue}{20}     &                        & shiftToRs2    \\
        23                     & constd & \textcolor{UniGrey}{W}    & \textcolor{UniBlue}{12}     &                        & shiftToFunct3 \\
        24                     & constd & \textcolor{UniGrey}{W}    & \textcolor{UniBlue}{25}     &                        & shiftToFunct7 \\
        25                     & constd & \textcolor{UniGrey}{W}    & \textcolor{UniBlue}{5}      &                        & shiftBy5      \\
        26                     & constd & \textcolor{UniGrey}{W}    & \textcolor{UniBlue}{11}     &                        & shiftBy11     \\
        \\
        27                     & constd & \textcolor{UniGrey}{W}    & \textcolor{UniBlue}{3}      &                        & load          \\
        28                     & constd & \textcolor{UniGrey}{W}    & \textcolor{UniBlue}{19}     &                        & opImm         \\
        29                     & constd & \textcolor{UniGrey}{W}    & \textcolor{UniBlue}{23}     &                        & auipc         \\
        30                     & constd & \textcolor{UniGrey}{W}    & \textcolor{UniBlue}{27}     &                        & opImm32       \\
        31                     & constd & \textcolor{UniGrey}{W}    & \textcolor{UniBlue}{35}     &                        & store         \\
        32                     & constd & \textcolor{UniGrey}{W}    & \textcolor{UniBlue}{51}     &                        & op            \\
        33                     & constd & \textcolor{UniGrey}{W}    & \textcolor{UniBlue}{55}     &                        & lui           \\
        34                     & constd & \textcolor{UniGrey}{W}    & \textcolor{UniBlue}{59}     &                        & op32          \\
        35                     & constd & \textcolor{UniGrey}{W}    & \textcolor{UniBlue}{99}     &                        & branch        \\
        36                     & constd & \textcolor{UniGrey}{W}    & \textcolor{UniBlue}{103}    &                        & jalr          \\
        37                     & constd & \textcolor{UniGrey}{W}    & \textcolor{UniBlue}{111}    &                        & jal           \\
        \hline
        \hline
    \end{tabular}
    \caption[Sorts and non-progressive Constants]{Sorts and non-progressive
        Constants for encoding RISC-V in BTOR2}\label{fig:constants}
\end{figure}


\begin{algorithm}
    \For{\textcolor{Green}{i} from 0 to 31}{
        add to model:\\
        \begin{tabular}[h]{>{\ttfamily\color{UniRed}}r >{\ttfamily}l >{\ttfamily\color{UniGrey}}l >{\slshape\color{UniRed}}l >{\slshape\color{UniRed}}l >{\slshape\color{UniRed}}l >{\slshape} l}
            \hline
            \hline
            n & constd & W & \texttt{\upshape\color{Green}i} &  &  & \textcolor{Green}{i}Const \\
            \hline
            \hline
        \end{tabular}\\
    }
    \caption[Progressive Constants]{Progressive Constants for encoding RISC-V in BTOR2}
    \label{alg:progressiveconsts}
\end{algorithm}

\subsection{State Representation}
The next logical step is defining a representation of a RISC-V state. Tis is
straightforward as shown in \figref{fig:states}. I also introduced a flag for
each register in my code. They track if the register was written to and makes
it possible to shorten a state file transformed from a witness to only the
relevant registers. As they have no impact on the operation of the BTOR2 model,
I will not mention them again. \begin{figure}
    %\centering
    \begin{center}

        \ttfamily
        %\begin{minipage}{.5\linewidth}
        \begin{tabular}[t]{>{\color{UniRed}}r l >{\color{UniGrey}}l l >{\slshape} l}
            \hline
            \hline
            (n + 0)  & state & D &  & x0  \\
            (n + 1)  & state & D &  & x1  \\
            (n + 2)  & state & D &  & x2  \\
            (n + 3)  & state & D &  & x3  \\
            (n + 4)  & state & D &  & x4  \\
            (n + 5)  & state & D &  & x5  \\
            (n + 6)  & state & D &  & x6  \\
            (n + 7)  & state & D &  & x7  \\
            (n + 8)  & state & D &  & x8  \\
            (n + 9)  & state & D &  & x9  \\
            (n + 10) & state & D &  & x10 \\
            (n + 11) & state & D &  & x11 \\
            (n + 12) & state & D &  & x12 \\
            (n + 13) & state & D &  & x13 \\
            (n + 14) & state & D &  & x14 \\
            (n + 15) & state & D &  & x15 \\
            (n + 16) & state & D &  & x16 \\
        \end{tabular}
        \hspace{1cm}
        %\end{minipage}%
        %\begin{minipage}{.5\linewidth}
        \begin{tabular}[t]{>{\color{UniRed}}r l >{\color{UniGrey}}l l >{\slshape} l}
            \\
            (n + 17) & state & D   &  & x17    \\
            (n + 18) & state & D   &  & x18    \\
            (n + 19) & state & D   &  & x19    \\
            (n + 20) & state & D   &  & x20    \\
            (n + 21) & state & D   &  & x21    \\
            (n + 22) & state & D   &  & x22    \\
            (n + 23) & state & D   &  & x23    \\
            (n + 24) & state & D   &  & x24    \\
            (n + 25) & state & D   &  & x25    \\
            (n + 26) & state & D   &  & x26    \\
            (n + 27) & state & D   &  & x27    \\
            (n + 28) & state & D   &  & x28    \\
            (n + 29) & state & D   &  & x29    \\
            (n + 30) & state & D   &  & x30    \\
            (n + 31) & state & D   &  & x31    \\
            (n + 32) & state & AS  &  & pc     \\
            (n + 33) & state & Mem &  & memory \\
            \hline
            \hline
        \end{tabular}
        %\end{minipage}
    \end{center}
    \caption[State representation for transforming RISC-V to BTOR2]{State representation for encoding}\label{fig:states}
\end{figure}



\subsection{Initialization}
To initialize a state in BTOR2 from a RISC-V state file, the values in the
registers must be loaded as constants, and for each memory address mentioned in
the state file, the value and address has to be loaded as constants. Due to the
inability to represent a full 64bit address space, the shrinking of the address
space from state file to BTOR2 model must be handled. I decided to just
initialiase the addresses up to the BTOR2 model address space maximum and cut
all others in the state file as I deem this the most predictable behaivour.
Everything not mentioned in the state file will be zero-initialised. At last
these constants must be used to initialise the state. For the registers this is
straight forward, for the memory we must first write all memory adresses into a
placeholder array wich then we can use to initialise the real memory. Due to
constraints in BTOR2, these constants have to be defined \textbf{before} the
states, but initialisation with the values must happen after the states. This
means that this initialisation process \textbf{wrappes around} the state
representation. The generation of constants is shown in
\algoref{alg:generateconstantsfromstate}, whereas the actual initialization is
shown in \algoref{alg:initstate}.
\begin{algorithm}
    $truePc \leftarrow$ value of pc in state file\\
    $maxPc \leftarrow$ number of addresses in BTOR2 model\\
    \textcolor{UniBlue}{pcValue} $\leftarrow truePc$ modulo $maxPc$\\
    add to model:\\
    \begin{tabular}[h]{>{\color{UniRed}}r l >{\color{UniGrey}}l l >{\slshape} l}
        \hline
        \hline
        \ttfamily
        (n + 0) & constd & AS & \textcolor{UniBlue}{\rmfamily\textsl{pcValue}} & pcConst \\
        \hline
        \hline
    \end{tabular}\\
    \textcolor{UniRed}{n} += 1
    \BlankLine

    \For{every register \textcolor{Green}{$x_i$}}{
        \If{register is initialised in state file}{
            \textcolor{UniBlue}{$registerValue$} $\leftarrow$ value of \textcolor{Green}{$x_i$}\\
            \If{\textcolor{UniBlue}{$registerValue$} $\neq 0$}{
                add to model:\\
                \begin{tabular}[h]{>{\color{UniRed}}r l >{\color{UniGrey}}l l >{\slshape} l}
                    \hline
                    \hline
                    \ttfamily
                    (n + 0) & constd & D & \textcolor{UniBlue}{\rmfamily$registerValue$} & \textcolor{Green}{$x_i$}Const \\
                    \hline
                    \hline
                \end{tabular}\\
                \textcolor{UniRed}{n} += 1
            }
        }
    }
    \BlankLine
    add to model:\\

    \begin{tabular}[h]{>{\ttfamily\color{UniRed}}r >{\ttfamily}l >{\ttfamily\color{UniGrey}}l >{\ttfamily}l >{\slshape} l}
        \hline
        \hline
        (n + 0) & state & Mem &                                     & memPH \\
        (n + 1) & init  & Mem & \textcolor{UniRed}{$memPH$ (n + 0)} &       \\
        \hline
        \hline
    \end{tabular}\\
    \textcolor{UniRed}{n} += 2\\

    $\textcolor{UniRed}{lastPH} \leftarrow \textcolor{UniRed}{memPH}$\\
    $allInitialCells \leftarrow$ all initialised memory cells in the state file\\
    $cutInitialCells \leftarrow$ remove all cells with address over maxPc\\
    \For{every cell \textcolor{Green}{$c$} in $cutInitialCells$}{
        $address \leftarrow$ address of \textcolor{Green}{$c$}\\
        $value \leftarrow$ value of \textcolor{Green}{$c$}\\
        add to model:\\
        \begin{tabular}[h]{>{\ttfamily\color{UniRed}}r >{\ttfamily}l >{\ttfamily\color{UniGrey}}l >{\ttfamily}l >{\slshape} l}
            \hline
            \hline
            \ttfamily
            (n + 0) & constd & AS  & \textcolor{UniBlue}{\rmfamily$address$}               &                             \\
            (n + 1) & constd & B   & \textcolor{UniBlue}{\rmfamily$value$}                 &                             \\
            (n + 2) & write  & Mem & \textcolor{UniRed}{\textrm{$lastPH$} (n + 0) (n + 1)} & PHAfter\textcolor{Green}{C} \\
            \hline
            \hline
        \end{tabular}\\
        \textcolor{UniRed}{n} += 3\\
        $\textcolor{UniRed}{lastPH} \leftarrow \textcolor{UniRed}{PHAfter\textcolor{Green}{C}}$\\
    }
    keep  $\textcolor{UniRed}{lastPH}$ for initialisation

    \caption[Generating initialisation constants]{Generating initialisation constants from state file in BTOR2}\label{alg:generateconstantsfromstate}
\end{algorithm}
\begin{algorithm}
    add to model:\\
    \begin{tabular}[h]{>{\ttfamily\color{UniRed}}r >{\ttfamily}l >{\ttfamily\color{UniGrey}}l >{\slshape\color{UniRed}}l >{\slshape} l}
        \hline
        \hline
        \ttfamily
        n & init & AS & pc\ \ pcConst & \\
        \hline
        \hline
    \end{tabular}\\
    \BlankLine
    \For{every register \textcolor{Green}{$x_i$}}{
        \If{\textcolor{Green}{$x_i$}\textcolor{UniRed}{Const} was defined}{
            add to model:\\
            \begin{tabular}[h]{>{\ttfamily\color{UniRed}}r >{\ttfamily}l >{\ttfamily\color{UniGrey}}l >{\slshape\color{UniRed}}l >{\slshape} l}
                \hline
                \hline
                \ttfamily
                n & init & D & \textcolor{Green}{$x_i$}\ \ \textcolor{Green}{$x_i$}Const & \\
                \hline
                \hline
            \end{tabular}\\
        }
    }
    \BlankLine
    With \textcolor{Green}{$lastPH$} from \algoref{alg:generateconstantsfromstate}\\
    add to model:\\
    \begin{tabular}[h]{>{\ttfamily\color{UniRed}}r >{\ttfamily}l >{\ttfamily\color{UniGrey}}l >{\slshape\color{UniRed}}l >{\slshape} l}
        \hline
        \hline
        \ttfamily
        n & init & Mem & memory\ \ \color{Green}lastPh & \\
        \hline
        \hline
    \end{tabular}\\
    \caption[Initialising states]{Initialising states in the BTOR2 model}\label{alg:initstate}
\end{algorithm}

\subsection{Fetching the current instruction}
To fetch the current instruction, i read the 4 bytes of the instruction and
concatenate them as seen in \figref{fig:fetching}
\begin{figure}
    \centering
    \begin{tabular}[h]{>{\ttfamily\color{UniRed}}r >{\ttfamily}l >{\ttfamily\color{UniGrey}}l >{\slshape\color{UniRed}}l >{\ttfamily\color{UniRed}}l >{\slshape} l}
        \hline
        \hline
        \ttfamily
        (n + 0) & add    & AS & addressInc               & \rmfamily\slshape pc &       \\
        (n + 1) & add    & AS & addressInc               & (n + 0)              &       \\
        (n + 2) & add    & AS & addressInc               & (n + 1)              &       \\\\
        (n + 3) & read   & B  & memory                   & \rmfamily\slshape pc &       \\
        (n + 4) & read   & B  & memory                   & (n + 0)              &       \\
        (n + 5) & read   & B  & memory                   & (n + 1)              &       \\
        (n + 6) & read   & B  & memory                   & (n + 2)              &       \\\\
        (n + 7) & concat & H  & \upshape\ttfamily(n + 4) & (n + 3)              &       \\
        (n + 8) & concat & H  & \upshape\ttfamily(n + 6) & (n + 5)              &       \\
        (n + 9) & concat & W  & \upshape\ttfamily(n + 8) & (n + 7)              & instr \\
        \hline
        \hline
    \end{tabular}
    \caption[Fetching instruction]{Fetching the current instruction from memory}\label{fig:fetching}
\end{figure}

\subsection{Deconstruction of the instruction}
Now having the instruction, we can deconstruct it to extract the $opcode$,
$rd$, $rs1$, $rs2$, $funct3$, $funct7$ and $imm$. For everything apart of
$imm$, this can be done by a shift and a masking. This is shown in
\figref{fig:extractNOimm}.

The immediate on the other hand must be first constructed from its subfields,
which can be referenced in \figref{fig:rv64i_formats}. In the BTOR2 model this
looks like in \figref{fig:extractimmbytype}. \todo{Reference to same method in
    riscvsim} There are three things i want to point out:\\ First, some of the
immediate subfields overlap exactly. I made use of this fact in lines (n + 1)
with the overlap of $imm[11:5]$ of I- and S-type, and (n + 21) with J- and
B-types $imm[10:5]$ overlap. Second, as described in \secref{sec:riscvIsa} the
immediate is always sign-extended. To archive this we make arithmetic right
shifts, which do sign extension for us and with this pull our highest immediate
bit to its correct place. Third, at line (n + 8), for sign extension we must
shift right by 19. As this matches the opcode for arithmetic instructions with
immediates, so I used this and did not create a new constant.

Now I have \textsl{iTypeImm}, \textsl{sTypeImm}, \textsl{bTypeImm},
\textsl{uTypeImm} and \textsl{jTypeImm}. But it would be easier to just have
one node \textsl{imm} where we can reference the immediate value regardless of
the instruction. This is done in \figref{fig:findingImm}, where first I defined
Bools wich check all opcodes that are neither R-type nor I-type. Then I chained
if-then-else nodes to catch instructions that are of J-type, U-Type, B-Type or
S-type. If the instruction is none of them, I can safely default to I-type as
R-type does not handle with an immediate value. At the end I extend $imm$ to
the 64bit RV64I demands.

At this point I can also extract the values of the designated $rs1$ and $rs2$
registers. I show this for $rs1$ in \figref{alg:extractrs1val}, it is the same
for $rs2$ except that the names must be changed to $rs2$. Also, the comparison
constants can be left out as they are already defined for $rs1$ and can be
referenced from there.

\begin{figure}
    \centering
    \begin{tabular}[h]{>{\ttfamily\color{UniRed}}r >{\ttfamily}l >{\ttfamily\color{UniGrey}}l >{\slshape\color{UniRed}}l >{\slshape\color{UniRed}}l >{\slshape} l}
        \hline
        \hline
        (n + 0) & and & W & instr                    & 7Bitmask      & opcode \\
        (n + 1) & srl & W & instr                    & shiftToRd     &        \\
        (n + 2) & and & W & \upshape\ttfamily(n + 1) & 5Bitmask      & rd     \\
        (n + 3) & srl & W & instr                    & shiftToRs1    &        \\
        (n + 4) & and & W & \upshape\ttfamily(n + 3) & 5Bitmask      & rs1    \\
        (n + 5) & srl & W & instr                    & shiftToRs2    &        \\
        (n + 6) & and & W & \upshape\ttfamily(n + 5) & 5Bitmask      & rs2    \\
        (n + 7) & srl & W & instr                    & shiftToFunct3 &        \\
        (n + 8) & and & W & \upshape\ttfamily(n + 7) & shiftRd       & funct3 \\
        (n + 9) & srl & W & instr                    & shiftToFunct7 & funct7 \\
        \hline
        \hline
    \end{tabular}
    \caption[Extraction (without immediate)]{Extraction of values from the instruction without $imm$}\label{fig:extractNOimm}
\end{figure}
\begin{figure}
    \centering
    \begin{tabular}[h]{>{\ttfamily\color{UniRed}}r >{\ttfamily}l >{\ttfamily\color{UniGrey}}l >{\slshape\color{UniRed}}l >{\slshape\color{UniRed}}l >{\slshape} l}
        \hline
        \hline
        (n + 0)  & sra & W & instr                     & shiftToRs2                                     & iTypeImm \\
        \\
        (n + 1)  & and & W & iTypeImm                  & \textcolor{Black}{\upshape\ttfamily-}5Bitmask  & s[11:5]  \\
        (n + 2)  & add & W & s[11:5]                   & rd                                             & sTypeImm \\
        \\
        (n + 3)  & and & W & rd                        & \textcolor{Black}{\upshape\ttfamily-}bitPicker & b[4:0]   \\
        (n + 4)  & and & W & funct7                    & 6Bitmask                                       &          \\
        (n + 5)  & sll & W & \upshape\ttfamily(n + 4)  & shiftBy5                                       & b[10:5]  \\
        (n + 6)  & and & W & bitPicker                 & rd                                             &          \\
        (n + 7)  & sll & W & \upshape\ttfamily(n + 6)  & shiftBy11                                      & b[11]    \\
        (n + 8)  & sra & W & instr                     & mathI                                          &          \\
        (n + 9)  & and & W & \upshape\ttfamily(n + 8)  & 12Bitmask                                      & b[31:12] \\
        (n + 10) & add & W & b[10:5]                   & b[4:0]                                         &          \\
        (n + 11) & add & W & b[11]                     & \upshape\ttfamily(n + 10)                      &          \\
        (n + 12) & add & W & b[31:12]                  & \upshape\ttfamily(n + 11)                      & bTypeImm \\
        \\
        (n + 13) & and & W & instr                     & \textcolor{Black}{\upshape\ttfamily-}12Bitmask & uTypeImm \\
        \\
        (n + 14) & and & W & rs2                       & \textcolor{Black}{\upshape\ttfamily-}bitPicker & j[4:0]   \\
        (n + 15) & and & W & rs2                       & bitPicker                                      &          \\
        (n + 16) & sll & W & \upshape\ttfamily(n + 15) & shiftBy11                                      & j[11]    \\
        (n + 17) & sll & W & funct3                    & shiftToFunct3                                  & j[14:12] \\
        (n + 18) & sll & W & rs1                       & shiftToRs1                                     & j[19:15] \\
        (n + 19) & sra & W & instr                     & shiftBy11                                      &          \\
        (n + 20) & and & W & \upshape\ttfamily(n + 19) & \textcolor{Black}{\upshape\ttfamily-}20Bitmask & j[31:20] \\
        (n + 21) & add & W & b[10:5]                   & j[4:0]                                         &          \\
        (n + 22) & add & W & j[11]                     & \upshape\ttfamily(n + 21)                      &          \\
        (n + 23) & add & W & j[14:12]                  & \upshape\ttfamily(n + 22)                      &          \\
        (n + 24) & add & W & j[19:15]                  & \upshape\ttfamily(n + 23)                      &          \\
        (n + 25) & add & W & j[31:20]                  & \upshape\ttfamily(n + 24)                      & jTypeImm \\
        \hline
        \hline
    \end{tabular}
    \caption[Extraction of the different $imm$ Types]{Extraction of all $imm$ Types from the Instruction}\label{fig:extractimmbytype}
\end{figure}

% ; Basics
% 1 sort bitvec 1 Bool
% 2 sort bitvec %d AS
% 3 sort bitvec 8 B
% 4 sort bitvec 16 H
% 5 sort bitvec 32 W
% 6 sort bitvec 64 D
% 7 sort array 2 3 Mem
% 8 zero 5 empty_reg
% 9 constd 5 31 register_bitmask
% 10 constd 5 7 shift_rd
% 11 constd 5 15 shift_rs1
% 12 constd 5 20 shift_rs2
% 13 constd 5 12 shift_funct3
% 14 constd 5 25 shift_funct7
% 15 one 5 bit_picker
% 16 one 1 true
% 17 zero 1 false
\begin{figure}
    \centering
    \begin{tabular}[h]{>{\ttfamily\color{UniRed}}r >{\ttfamily}l >{\ttfamily\color{UniGrey}}l >{\slshape\color{UniRed}}l >{\slshape\color{UniRed}}l >{\slshape\color{UniRed}}l >{\slshape} l}
        \hline
        \hline
        (n + 0) & eq  & Bool & opcode  & store    &          & isSType \\
        (n + 1) & eq  & Bool & opcode  & branch   &          & isBType \\
        (n + 2) & eq  & Bool & opcode  & auipc    &          & uType1  \\
        (n + 3) & eq  & Bool & opcode  & lui      &          & uType2  \\
        (n + 4) & or  & Bool & uType1  & uType2   &          & isUType \\
        (n + 5) & eq  & Bool & opcode  & jump     &          & isJType \\
        \\
        (n + 6) & ite & W    & isSType & sTypeImm & iTypeImm & checkS  \\
        (n + 7) & ite & W    & isBType & bTypeImm & checkS   & checkB  \\
        (n + 8) & ite & W    & isUType & uTypeImm & checkB   & checkU  \\
        (n + 9) & ite & W    & isJType & jTypeImm & checkU   & imm     \\
        \hline
        \hline
    \end{tabular}
    \caption[Finding the correct immediate]{Finding the correct immediate by opcode}\label{fig:findingImm}
\end{figure}
\begin{algorithm}
    \For{\textcolor{Green}{i} from 1 to 31}{
        add to model:\\
        \begin{tabular}[h]{>{\ttfamily\color{UniRed}}r >{\ttfamily}l >{\ttfamily\color{UniGrey}}l >{\slshape\color{UniRed}}l >{\slshape\color{UniRed}}l >{\slshape\color{UniRed}}l >{\slshape} l}
            \hline
            \hline
            n & eq & Bool & rs1 & \textcolor{Green}{i}Const &  & isRs1X\textcolor{Green}{i} \\
            \hline
            \hline
        \end{tabular}\\
    }
    add to model:\\
    \begin{tabular}[h]{>{\ttfamily\color{UniRed}}r >{\ttfamily}l >{\ttfamily\color{UniGrey}}l >{\slshape\color{UniRed}}l >{\slshape\color{UniRed}}l >{\slshape\color{UniRed}}l >{\slshape} l}
        \hline
        \hline
        n & ite & D & isRs1X1 & x1 & x0 & checkX1 \\
        \hline
        \hline
    \end{tabular}\\
    \BlankLine
    \For{\textcolor{Green}{i} from 2 to 30}{
        add to model:\\
        \begin{tabular}[h]{>{\ttfamily\color{UniRed}}r >{\ttfamily}l >{\ttfamily\color{UniGrey}}l >{\slshape\color{UniRed}}l >{\slshape\color{UniRed}}l >{\slshape\color{UniRed}}l >{\slshape} l}
            \hline
            \hline
            n & ite & D & isRs1X\textcolor{Green}{i} & x\textcolor{Green}{i} & checkX\textcolor{Green}{(i - 1)} & checkX\textcolor{Green}{i} \\
            \hline
            \hline
        \end{tabular}\\
    }
    add to model:\\
    \begin{tabular}[h]{>{\ttfamily\color{UniRed}}r >{\ttfamily}l >{\ttfamily\color{UniGrey}}l >{\slshape\color{UniRed}}l >{\slshape\color{UniRed}}l >{\slshape\color{UniRed}}l >{\slshape} l}
        \hline
        \hline
        n & ite & D & isRs1X31 & x31 & checkX30 & rs1val \\
        \hline
        \hline
    \end{tabular}\\

    \caption[Value extraction from $rs1$]{Extracting the value of the register designated by $rs1$}
    \label{alg:extractrs1val}
\end{algorithm}

\subsection{Instruction Detection}

For the next-state logic, the only thing left that we need to know is the
actual current command. So I defined a check
\textsl{is\textcolor{Green}{Instruction}} for each instruction. As this is
quite repetetive, \algoref{alg:commanddetection} describes a generalised
approach to reach these Bools. An example for each instruction subgroup in
\algoref{alg:commanddetection} can be found in \figref{fig:detectionexample}.
Of course the funct7 checks from the $needsf7$ subgroup can be reused if
multiple instructions use the same funct7.

\begin{algorithm}
    add to model:\\
    \begin{tabular}[h]{>{\ttfamily\color{UniRed}}r >{\ttfamily}l >{\ttfamily\color{UniGrey}}l >{\slshape\color{UniRed}}l >{\slshape\color{UniRed}}l >{\slshape\color{UniRed}}l >{\slshape} l}
        \hline
        \hline
        (n + 0)  & eq & Bool & opcode & load    &  & isLoad    \\
        (n + 1)  & eq & Bool & opcode & opImm   &  & isOpImm   \\
        (n + 2)  & eq & Bool & opcode & auipc   &  & isAUIPC   \\
        (n + 3)  & eq & Bool & opcode & opImm32 &  & isOpImm32 \\
        (n + 4)  & eq & Bool & opcode & store   &  & isStore   \\
        (n + 5)  & eq & Bool & opcode & op      &  & isOp      \\
        (n + 6)  & eq & Bool & opcode & lui     &  & isLUI     \\
        (n + 7)  & eq & Bool & opcode & op32    &  & isOp32    \\
        (n + 8)  & eq & Bool & opcode & branch  &  & isBranch  \\
        (n + 9)  & eq & Bool & opcode & jalr    &  & isJALR    \\
        (n + 10) & eq & Bool & opcode & jal     &  & isJAL     \\
        \hline
        \hline
    \end{tabular}\\
    \textcolor{UniRed}{n} += 11\\
    \For{\textcolor{Green}{i} from 0 to 7}{
        add to model:\\
        \begin{tabular}[h]{>{\ttfamily\color{UniRed}}r >{\ttfamily}l >{\ttfamily\color{UniGrey}}l >{\slshape\color{UniRed}}l >{\slshape\color{UniRed}}l >{\slshape\color{UniRed}}l >{\slshape} l}
            \hline
            \hline
            (n + 0) & constd & W    & \texttt{\upshape\color{Green}i} &                           &  & \textcolor{Green}{i}Const    \\
            (n + 1) & eq     & Bool & funct3                          & \textcolor{Green}{i}Const &  & is\textcolor{Green}{i}Funct3 \\
            \hline
            \hline
        \end{tabular}\\
        \textcolor{UniRed}{n} += 2\\
    }
    $onlyOp \leftarrow$ [LUI, AUIPC, JAL, JALR]\\
    $needsf7 \leftarrow$ [SRL, SRA, SRLI, SRAI, SRLW, SRAW, SRLWI, SRAWI, ADD, SUB, ADDW, SUBW]\\
    $rest \leftarrow$[ all other instructions ]\\
    \BlankLine
    \For{ all instructions \textcolor{Green}{I} in $onlyOp$}{
        $is\textcolor{Green}{I}$ is already defined
    }
    \For{all instructions  \textcolor{Green}{I} in $rest$ }{
        $opname \leftarrow$ opcode name of \textcolor{Green}{I}\\
        $f3val \leftarrow$ expected funct3 of \textcolor{Green}{I} as digit\\
        add to model:\\
        \begin{tabular}[h]{>{\ttfamily\color{UniRed}}r >{\ttfamily}l >{\ttfamily\color{UniGrey}}l >{\slshape\color{UniRed}}l >{\slshape\color{UniRed}}l >{\slshape\color{UniRed}}l >{\slshape} l}
            \hline
            \hline
            n & and & Bool & is\textcolor{Black}{opname} & is\textcolor{Black}{f3val}Funct3 &  & is\textcolor{Green}{I} \\
            \hline
            \hline
        \end{tabular}\\
        \textcolor{UniRed}{n} += 1\\
    }
    \For{all instructions  \textcolor{Green}{I} in $needsf7$ }{
        $opname \leftarrow$ opcode name of \textcolor{Green}{I}\\
        $f3val \leftarrow$ expected funct3 of \textcolor{Green}{I} as digit\\
        $f7hex \leftarrow$ expected funct7 of \textcolor{Green}{I} as hexadecimal number\\
        add to model:\\
        \begin{tabular}[h]{>{\ttfamily\color{UniRed}}r >{\ttfamily}l >{\ttfamily\color{UniGrey}}l >{\slshape\color{UniRed}}l >{\slshape\color{UniRed}}l >{\slshape\color{UniRed}}l >{\slshape} l}
            \hline
            \hline
            (n + 0) & consth & W    & \textcolor{UniBlue}{f7hex}       &                            &  & \textcolor{Green}{I}f7     \\
            (n + 1) & eq     & Bool & funct7                           & \textcolor{Green}{I}f7     &  & fitsF7\textcolor{Green}{I} \\
            (n + 2) & and    & Bool & is\textcolor{Black}{f3val}Funct3 & fitsF7\textcolor{Green}{I} &  & fitsF3\textcolor{Green}{I} \\
            (n + 3) & and    & Bool & is\textcolor{Black}{opname}      & fitsF3\textcolor{Green}{I} &  & is\textcolor{Green}{I}     \\
            \hline
            \hline
        \end{tabular}\\
        \textcolor{UniRed}{n} += 4\\
    }
    \caption[Instruction detection]{Generalised approach to instruction detection}\label{alg:commanddetection}
\end{algorithm}
\begin{figure}
    \centering
    \begin{tabular}[h]{>{\ttfamily\color{UniRed}}r >{\ttfamily}l >{\ttfamily\color{UniGrey}}l >{\slshape\color{UniRed}}l >{\slshape\color{UniRed}}l >{\slshape\color{UniRed}}l >{\slshape} l}
        \hline
        \hline
        \multicolumn{7}{l}{\rmfamily(\slshape isJALR \upshape already exists)}                   \\
        \hline
        n\ \    & and    & Bool & isLoad                  & is5Funct3                &  & isLHU  \\
        \hline
        (n + 0) & consth & W    & \textcolor{UniBlue}{20} &                          &  &        \\
        (n + 1) & eq     & Bool & funct7                  & \upshape\ttfamily(n + 0) &  &        \\
        (n + 2) & and    & Bool & is0Funct3               & \upshape\ttfamily(n + 1) &  &        \\
        (n + 3) & and    & Bool & isLoad                  & \upshape\ttfamily(n + 2) &  & isSUBW \\
        \hline
        \hline
    \end{tabular}
    \caption[Examples for Instruction Detection]{Instruction Detection of \texttt{JALR}, \texttt{LHU} and \texttt{SUBW} as described in \algoref{alg:commanddetection}}\label{fig:detectionexample}
\end{figure}

\subsection{Next-State Logic}
The next state logic is basically the core of the model. Almost everything else
works towards this point. The Goal is to create the changes each instruction
would make and then only inserting the changes specific to the instruction in
the state. Each state node in the model must have an accompanying next node to
work as intended. But first the changed values are needed.

\subsection{Creating all Values of Instruction execution}
It would be too long and unnecessary to go through all instructions, as this is
simply following the RV64I ISA, but I want to give an example for each group of
instructions as they were divided in \tabref{tab:rv64i-instructions}. I show
this for \texttt{AUIPC}, \texttt{JALR}, \texttt{BEQ}, \texttt{LHU},
\texttt{SD}, \texttt{ANDI}, \texttt{SLLIW}, \texttt{SLT} and \texttt{SUBW} in
\figref{fig:valueexample}. In this examples one can see multiple overlaps which
can be used, e.g. the adresses for load and store instructions or the 32bit
versions of the word instructions. Also I took \texttt{SD} to show that all
other store instructions happen as interim results of preparing \texttt{SD}. It
is similar with load instructions, but here we only get overlapping pre-results
which each have to be sign extended to the expected 64bit on their own.

With this done we can sort each change to its instruction.

\begin{figure}%
    \centering

    \subfloat[AUIPC]{
        \begin{tabular}[h]{>{\ttfamily\color{UniRed}}r >{\ttfamily}l >{\ttfamily\color{UniGrey}}l >{\slshape\color{UniRed}}l >{\slshape\color{UniRed}}l >{\slshape\color{UniRed}}l >{\slshape} l}
            \hline
            \hline
            (n + 0) & add & D & imm & pc &  & rdAUIPC \\
            \hline
            \hline
        \end{tabular}
    }

    %    \caption[Examples for instruction execution]{Instruction execution for chosen instructions}\label{fig:valueexample}
    %\end{figure}
    %\begin{figure}%
    %    \ContinuedFloat
    %    \centering

    \subfloat[JALR]{
        \begin{tabular}[h]{>{\ttfamily\color{UniRed}}r >{\ttfamily}l >{\ttfamily\color{UniGrey}}l >{\slshape\color{UniRed}}l >{\slshape\color{UniRed}}l >{\slshape\color{UniRed}}l >{\slshape} l}
            \hline
            \hline
            (n + 0) & add   & AS & pc                         & pcInc                   &  & nextPc      \\
            (n + 0) & add   & D  & imm                        & rs1val                  &  & pcJALR64pre \\
            (n + 0) & and   & D  & \textcolor{Black}{-}1Const & pcJALR64pre             &  & pcJALR64    \\
            (n + 0) & slice & AS & pcJALR64                   & \textcolor{UniBlue}{15} &  & pcJALR      \\
            (n + 0) & uext  & D  & nextPc                     & \textcolor{UniBlue}{48} &  & rdJALR      \\
            \hline
            \hline
            \multicolumn{7}{l}{\todo{pc overflow erwähnen}}
        \end{tabular}
    }

    %    \caption[]{Instruction execution for chosen instructions}\label{fig:valueexample}
    %\end{figure}
    %\begin{figure}%
    %    \ContinuedFloat
    %    \centering

    \subfloat[BEQ]{
        \begin{tabular}[h]{>{\ttfamily\color{UniRed}}r >{\ttfamily}l >{\ttfamily\color{UniGrey}}l >{\slshape\color{UniRed}}l >{\slshape\color{UniRed}}l >{\slshape\color{UniRed}}l >{\slshape} l}
            \hline
            \hline
            (n + 0) & add   & AS   & pc        & pcInc                   &                        & nextPc    \\
            (n + 0) & slice & AS   & imm       & \textcolor{UniBlue}{15} & \textcolor{UniBlue}{0} & branchImm \\
            (n + 0) & add   & AS   & pc        & branchImm               &                        & branchPc  \\
            (n + 0) & eq    & Bool & rs1val    & rs2val                  &                        & isBEQcond \\
            (n + 0) & ite   & AS   & isBEQcond & branchPc                & nextPc                 & pcBEQ     \\
            \hline
            \hline
        \end{tabular}
    }

    %    \caption[]{Instruction execution for chosen instructions}\label{fig:valueexample}
    %\end{figure}
    %\begin{figure}%
    %    \ContinuedFloat
    %    \centering

    \subfloat[LHU]{
        \begin{tabular}[h]{>{\ttfamily\color{UniRed}}r >{\ttfamily}l >{\ttfamily\color{UniGrey}}l >{\slshape\color{UniRed}}l >{\slshape\color{UniRed}}l >{\slshape\color{UniRed}}l >{\slshape} l}
            \hline
            \hline
            (n + 0) & add    & D  & rs1val     & imm                     &                        & 1stAddrPre \\
            (n + 0) & slice  & AS & 1stAddrPre & \textcolor{UniBlue}{15} & \textcolor{UniBlue}{0} & 1stAddr    \\
            (n + 0) & add    & AS & 1stAddr    & addressInc              &                        & 2ndAddr    \\
            (n + 0) & read   & B  & memory     & 1stAddr                 &                        & loadB1     \\
            (n + 0) & read   & B  & memory     & 2ndAddr                 &                        & loadB2     \\
            (n + 0) & concat & H  & loadB2     & loadB1                  &                        & loadB2B1   \\
            (n + 0) & uext   & D  & loadB2B1   & \textcolor{UniBlue}{48} & \textcolor{UniBlue}{0} & rdLHU      \\
            \hline
            \hline
        \end{tabular}
    }

    %    \caption[]{Instruction execution for chosen instructions}\label{fig:valueexample}
    %\end{figure}
    %\begin{figure}%
    %    \ContinuedFloat
    %    \centering

    \subfloat[ANDI]{
        \begin{tabular}[h]{>{\ttfamily\color{UniRed}}r >{\ttfamily}l >{\ttfamily\color{UniGrey}}l >{\slshape\color{UniRed}}l >{\slshape\color{UniRed}}l >{\slshape\color{UniRed}}l >{\slshape} l}
            \hline
            \hline
            (n + 0) & and & D & rs1val & imm &  & rdANDI \\
            \hline
            \hline
        \end{tabular}
    }

    %    \caption[]{Instruction execution for chosen instructions}\label{fig:valueexample}
    %\end{figure}
    %\begin{figure}%
    %    \ContinuedFloat
    %    \centering

    \subfloat[SLT]{
        \begin{tabular}[h]{>{\ttfamily\color{UniRed}}r >{\ttfamily}l >{\ttfamily\color{UniGrey}}l >{\slshape\color{UniRed}}l >{\slshape\color{UniRed}}l >{\slshape\color{UniRed}}l >{\slshape} l}
            \hline
            \hline
            (n + 0) & slt  & Bool & rs1val   & rs2val                  &  & rdSLTpre \\
            (n + 0) & uext & D    & rdSLTpre & \textcolor{UniBlue}{63} &  & rdSLT    \\
            \hline
            \hline
        \end{tabular}
    }

    \caption[]{Instruction execution for chosen instructions}\label{fig:valueexample}
\end{figure}
\begin{figure}%
    \ContinuedFloat
    \centering

    \subfloat[SD]{
        \begin{tabular}[h]{>{\ttfamily\color{UniRed}}r >{\ttfamily}l >{\ttfamily\color{UniGrey}}l >{\slshape\color{UniRed}}l >{\slshape\color{UniRed}}l >{\slshape\color{UniRed}}l >{\slshape} l}
            \hline
            \hline
            (n + 0) & add   & D   & rs1val     & imm                     &                         & 1stAddrPre \\
            (n + 0) & slice & AS  & 1stAddrPre & \textcolor{UniBlue}{15} & \textcolor{UniBlue}{0}  & 1stAddr    \\
            (n + 0) & add   & AS  & 1stAddr    & addressInc              &                         & 2ndAddr    \\
            (n + 0) & add   & AS  & 2ndAddr    & addressInc              &                         & 3rdAddr    \\
            (n + 0) & add   & AS  & 3rdAddr    & addressInc              &                         & 4thAddr    \\
            (n + 0) & add   & AS  & 4thAddr    & addressInc              &                         & 5thAddr    \\
            (n + 0) & add   & AS  & 5thAddr    & addressInc              &                         & 6thAddr    \\
            (n + 0) & add   & AS  & 6thAddr    & addressInc              &                         & 7thAddr    \\
            (n + 0) & add   & AS  & 7thAddr    & addressInc              &                         & 8thAddr    \\
            \\
            (n + 0) & slice & B   & rs2val     & \textcolor{UniBlue}{7}  & \textcolor{UniBlue}{0}  & storeB1    \\
            (n + 0) & slice & B   & rs2val     & \textcolor{UniBlue}{15} & \textcolor{UniBlue}{8}  & storeB2    \\
            (n + 0) & slice & B   & rs2val     & \textcolor{UniBlue}{23} & \textcolor{UniBlue}{16} & storeB3    \\
            (n + 0) & slice & B   & rs2val     & \textcolor{UniBlue}{31} & \textcolor{UniBlue}{24} & storeB4    \\
            (n + 0) & slice & B   & rs2val     & \textcolor{UniBlue}{39} & \textcolor{UniBlue}{32} & storeB5    \\
            (n + 0) & slice & B   & rs2val     & \textcolor{UniBlue}{47} & \textcolor{UniBlue}{40} & storeB6    \\
            (n + 0) & slice & B   & rs2val     & \textcolor{UniBlue}{55} & \textcolor{UniBlue}{48} & storeB7    \\
            (n + 0) & slice & B   & rs2val     & \textcolor{UniBlue}{63} & \textcolor{UniBlue}{56} & storeB8    \\
            \\
            (n + 0) & write & Mem & memory     & 1stAddr                 & storeB1                 & memorySB   \\
            (n + 0) & write & Mem & memorySB   & 2ndAddr                 & storeB2                 & memorySH   \\
            (n + 0) & write & Mem & memorySH   & 3rdAddr                 & storeB3                 & memoryB3   \\
            (n + 0) & write & Mem & memoryB3   & 4thAddr                 & storeB4                 & memorySW   \\
            (n + 0) & write & Mem & memorySW   & 5thAddr                 & storeB5                 & memoryB5   \\
            (n + 0) & write & Mem & memoryB5   & 6thAddr                 & storeB6                 & memoryB6   \\
            (n + 0) & write & Mem & memoryB6   & 7thAddr                 & storeB7                 & memoryB7   \\
            (n + 0) & write & Mem & memoryB7   & 8thAddr                 & storeB8                 & memorySD   \\
            \hline
            \hline
        \end{tabular}
    }

    %    \caption[]{Instruction execution for chosen instructions}\label{fig:valueexample}
    %\end{figure}
    %\begin{figure}%
    %    \ContinuedFloat
    %    \centering

    \subfloat[SLLIW]{
        \begin{tabular}[h]{>{\ttfamily\color{UniRed}}r >{\ttfamily}l >{\ttfamily\color{UniGrey}}l >{\slshape\color{UniRed}}l >{\slshape\color{UniRed}}l >{\slshape\color{UniRed}}l >{\slshape} l}
            \hline
            \hline
            (n + 0) & and   & W & imm32    & 5Bitmask                &                        & shamtIW    \\
            (n + 0) & slice & W & rs1val   & \textcolor{UniBlue}{31} & \textcolor{UniBlue}{0} & rs1val32   \\
            (n + 0) & sll   & W & rs1val32 & shamtIW                 &                        & rdSLLIWpre \\
            (n + 0) & sext  & D & rs1val32 & 32                      &                        & rdSLLIW    \\
            \hline
            \hline
        \end{tabular}
    }

    %    \caption[]{Instruction execution for chosen instructions}\label{fig:valueexample}
    %\end{figure}
    %\begin{figure}%
    %    \ContinuedFloat
    %    \centering

    \subfloat[SUBW]{
        \begin{tabular}[h]{>{\ttfamily\color{UniRed}}r >{\ttfamily}l >{\ttfamily\color{UniGrey}}l >{\slshape\color{UniRed}}l >{\slshape\color{UniRed}}l >{\slshape\color{UniRed}}l >{\slshape} l}
            \hline
            \hline
            (n + 0) & slice & W & rs1val    & \textcolor{UniBlue}{31} & \textcolor{UniBlue}{0} & rs1val32  \\
            (n + 0) & slice & W & rs2val    & \textcolor{UniBlue}{31} & \textcolor{UniBlue}{0} & rs2val32  \\
            (n + 0) & sub   & W & rs1val32  & rs2val32                &                        & rdSUBWpre \\
            (n + 0) & sext  & D & rdSUBWpre & \textcolor{UniBlue}{32} &                        & rdSUBW    \\
            \hline
            \hline
        \end{tabular}
    }

    \caption[]{Instruction execution for chosen instructions}\label{fig:valueexample}
\end{figure}

\subsubsection{The next Memory}
Defining the next memory array is simple. I just cascade through all store
instructions with if-then-else nodes and by setting the final 'else' as the
current memory array, if no 'if' catches, the array is not changed. All this is
shown in \figref{fig:nextmemory}.

\begin{figure}
    \centering
    \begin{tabular}[h]{>{\ttfamily\color{UniRed}}r >{\ttfamily}l >{\ttfamily\color{UniGrey}}l >{\slshape\color{UniRed}}l >{\slshape\color{UniRed}}l >{\slshape\color{UniRed}}l >{\slshape} l}
        \hline
        \hline
        (n + 0) & ite  & Mem & isSB   & memorySB & memory  & newMem3 \\
        (n + 1) & ite  & Mem & isSH   & memorySH & newMem3 & newMem2 \\
        (n + 2) & ite  & Mem & isSW   & memorySW & newMem2 & newMem1 \\
        (n + 3) & ite  & Mem & isSD   & memorySD & newMem1 & newMem  \\
        (n + 4) & next & Mem & memory & newMem                       \\
        \hline
        \hline
    \end{tabular}
    \caption[Next-State logic for memory]{Next-State logic for the memory array}\label{fig:nextmemory}
\end{figure}

\subsubsection{The next pc}
For the next pc it looks mostly the same as shown in \figref{fig:nextpc}. Only
the behaivour if no 'if' catches is different as pc must point to the next
instruction to execute. This nextPc was already computed for the JAL and JALR
instructions so I reused it. The unconditional jumps also change the value in
rd, but this is done in the next subsection.

\begin{figure}
    \centering
    \begin{tabular}[h]{>{\ttfamily\color{UniRed}}r >{\ttfamily}l >{\ttfamily\color{UniGrey}}l >{\slshape\color{UniRed}}l >{\slshape\color{UniRed}}l >{\slshape\color{UniRed}}l >{\slshape} l}
        \hline
        \hline
        (n + 0) & ite  & AS & isBGEU & pcBGEU                   & nextPc                   & \\
        (n + 1) & ite  & AS & isBLTU & pcBLTU                   & \upshape\ttfamily(n + 0) & \\
        (n + 2) & ite  & AS & isBGE  & pcBGE                    & \upshape\ttfamily(n + 1) & \\
        (n + 3) & ite  & AS & isBLT  & pcBLT                    & \upshape\ttfamily(n + 2) & \\
        (n + 4) & ite  & AS & isBNE  & pcBNE                    & \upshape\ttfamily(n + 3) & \\
        (n + 5) & ite  & AS & isBEQ  & pcBEQ                    & \upshape\ttfamily(n + 4) & \\
        (n + 6) & ite  & AS & isJALR & pcJALR                   & \upshape\ttfamily(n + 5) & \\
        (n + 7) & ite  & AS & isJAL  & pcJAL                    & \upshape\ttfamily(n + 6) & \\
        (n + 8) & next & AS & pc     & \upshape\ttfamily(n + 7) &                          & \\
        \hline
        \hline
    \end{tabular}
    \caption[Next-State Logic for $pc$]{Next-State Logic for the $pc$ Register}\label{fig:nextpc}
\end{figure}

\subsubsection{The next rd}
At last the x registers must be updated. The procedure is defined in
\figref{alg:nextrd}. Whith exeption of x0 this is the same for all these
registers. Also it is similar in its procedure as defining the next memory or
pc but instead of a hand full of instructions, I have to go over 39 of them as
only branch and store instructions do not change rd. Because of this, I took
the liberty to not exactly show the cascade for all relevant instructions in
\algoref{alg:nextrd} but only indicate it.

\begin{algorithm}
    add to model:\\
    \begin{tabular}[h]{>{\ttfamily\color{UniRed}}r >{\ttfamily}l >{\ttfamily\color{UniGrey}}l >{\slshape\color{UniRed}}l >{\slshape\color{UniRed}}l >{\slshape\color{UniRed}}l >{\slshape} l}
        \hline
        \hline
        n & next & D & x0 & x0 \\
        \hline
        \hline
    \end{tabular}\\

    \For{\textcolor{Green}{i} from 1 to 31}{ add to model:
        \begin{tabular}[h]{>{\ttfamily\color{UniRed}}r >{\ttfamily}l >{\ttfamily\color{UniGrey}}l >{\slshape\color{UniRed}}l >{\slshape\color{UniRed}}l >{\slshape\color{UniRed}}l >{\slshape} l}
            \hline
            \hline
            (n + 0)  & ite  & D    & isLUI                     & rdLUI                     & x\textcolor{Green}{i}     & \\
            (n + 1)  & ite  & D    & isAUIPC                   & rdAUIPC                   & \upshape\ttfamily(n + 0)  & \\
            \vdots   & ite  & D    & \vdots                    & \vdots                    & \vdots                    & \\
            (n + 47) & ite  & D    & isSRAW                    & rdSRAW                    & \upshape\ttfamily(n + 46) & \\
            \\
            (n + 48) & eq   & Bool & rd                        & \textcolor{Green}{i}Const &                           & \\
            (n + 49) & ite  & D    & \upshape\ttfamily(n + 48) & \upshape\ttfamily(n + 47) & x\textcolor{Green}{i}     & \\
            (n + 50) & next & D    & x\textcolor{Green}{i}     & \upshape\ttfamily(n + 49)                               \\
            \hline
            \hline
        \end{tabular}\\
    }
    \caption[Next-State Logic for $rd$]{Next-State Logic for all Registers but $pc$}\label{alg:nextrd}
\end{algorithm}

\subsection{Constraints}
The only thing left is to define constraints to end the model checker. First is
the intended end of reaching a set number of Iterations. It is shown in
\figref{fig:badcounter}.

After this I defined some extra constraints to check for bad instructions.
First is checked if the opcode is valid for my model. The second constraint
catches if the instruction can not be detected even whilst the opcode is valid.
This is shown in \figref{fig:unknownInstr}. The constraint in
\figref{fig:badaddress} handles instruction-address-misaligned exceptions for
jump instructions.

Of course other constraints can be defined. Options would be to stop on a
specific pc or if a register reaches a specified value.

\todo{Maybe add examples on how to do?}\\
\todo{Maybe internal references in figures should be numbers\dots}

\begin{figure}
    \centering
    \begin{tabular}[h]{>{\ttfamily\color{UniRed}}r >{\ttfamily}l >{\ttfamily\color{UniGrey}}l >{\slshape\color{UniRed}}l >{\slshape\color{UniRed}}l >{\slshape\color{UniRed}}l >{\slshape} l}
        \hline
        \hline
        (n + 0) & one    & D    &                                &                          &  &         \\
        (n + 1) & constd & D    & \textcolor{Green}{nIterations} &                          &  &         \\
        (n + 2) & state  & D    &                                &                          &  & counter \\
        (n + 3) & init   & D    & \upshape\ttfamily(n + 2)       & emptyReg                              \\
        (n + 4) & add    & D    & \upshape\ttfamily(n + 2)       & \upshape\ttfamily(n + 0) &  &         \\
        (n + 5) & next   & D    & \upshape\ttfamily(n + 2)       & \upshape\ttfamily(n + 4)              \\
        (n + 6) & eq     & Bool & \upshape\ttfamily(n + 2)       & \upshape\ttfamily(n + 1) &  &         \\
        (n + 7) & bad    &      & \upshape\ttfamily(n + 6)                                               \\
        \hline
        \hline
    \end{tabular}
    \caption[Iterations counter property]{Terminating the model by iteration count}\label{fig:badcounter}
\end{figure}
\begin{figure}
    \centering
    \begin{tabular}[h]{>{\ttfamily\color{UniRed}}r >{\ttfamily}l >{\ttfamily\color{UniGrey}}l >{\slshape\color{UniRed}}l >{\slshape\color{UniRed}}l >{\slshape\color{UniRed}}l >{\slshape} l}
        \hline
        \hline
        (n + 0)  & or  & Bool & isLoad                                       & isOpImm                   &  & \\
        (n + 1)  & or  & Bool & isAUIPC                                      & \upshape\ttfamily(n + 0)  &  & \\
        (n + 2)  & or  & Bool & isOpImm32                                    & \upshape\ttfamily(n + 1)  &  & \\
        (n + 3)  & or  & Bool & isStore                                      & \upshape\ttfamily(n + 2)  &  & \\
        (n + 4)  & or  & Bool & isOp                                         & \upshape\ttfamily(n + 3)  &  & \\
        (n + 5)  & or  & Bool & isLUI                                        & \upshape\ttfamily(n + 4)  &  & \\
        (n + 6)  & or  & Bool & isOp32                                       & \upshape\ttfamily(n + 5)  &  & \\
        (n + 7)  & or  & Bool & isBranch                                     & \upshape\ttfamily(n + 6)  &  & \\
        (n + 8)  & or  & Bool & isJALR                                       & \upshape\ttfamily(n + 7)  &  & \\
        (n + 9)  & or  & Bool & isJAL                                        & \upshape\ttfamily(n + 8)  &  & \\
        (n + 10) & bad &      & \upshape\ttfamily\textcolor{Black}{-}(n + 9)                                  \\
        \\
        (n + 11) & or  & Bool & isLUI                                        & isAUIPC                   &  & \\
        (n + 12) & or  & Bool & isJAL                                        & \upshape\ttfamily(n + 11) &  & \\
        \vdots   & or  & Bool & \vdots                                       & \vdots                    &  & \\
        (n + 58) & or  & Bool & isSRAW                                       & \upshape\ttfamily(n + 57) &  & \\
        (n + 59) & and & Bool & \upshape\ttfamily\textcolor{Black}{-}(n + 9) & \upshape\ttfamily(n + 58) &  & \\
        (n + 60) & bad &      & \upshape\ttfamily(n + 59)                                                     \\
        \hline
        \hline
    \end{tabular}
    \caption[Unknown Instruction properties]{Terminating the model on unknown instructions}\label{fig:unknownInstr}
\end{figure}
\begin{figure}
    \centering
    \begin{tabular}[h]{>{\ttfamily\color{UniRed}}r >{\ttfamily}l >{\ttfamily\color{UniGrey}}l >{\slshape\color{UniRed}}l >{\slshape\color{UniRed}}l >{\slshape\color{UniRed}}l >{\slshape} l}
        \hline
        \hline
        (n + 0) & zero   & AS   &                  &                  &  & pcZero           \\
        (n + 0) & constd & AS   & 3                &                  &  & pcBitmask        \\
        (n + 0) & and    & AS   & pcBitmask        & pcJAL            &  & lowbitsJAL       \\
        (n + 0) & and    & AS   & pcBitmask        & pcJALR           &  & lowbitsJALR      \\
        (n + 0) & and    & AS   & pcBitmask        & pcBEQ            &  & lowbitsBEQ       \\
        \vdots  & and    & AS   & pcBitmask        & \vdots           &  & \vdots           \\
        (n + 0) & and    & AS   & pcBitmask        & pcBGEU           &  & lowbitsBGEU      \\
        \\
        (n + 0) & neq    & Bool & pcZero           & lowbitsJAL       &  & pcMisalignedJAL  \\
        (n + 0) & neq    & Bool & pcZero           & lowbitsJALR      &  & pcMisalignedJALR \\
        (n + 0) & neq    & Bool & pcZero           & lowbitsBEQ       &  & pcMisalignedBEQ  \\
        \vdots  & neq    & Bool & pcZero           & \vdots           &  & \vdots           \\
        (n + 0) & neq    & Bool & pcZero           & lowbitsBGEU      &  & pcMisalignedBGEU \\
        \\
        (n + 0) & or     & Bool & pcMisalignedJAL  & pcMisalignedJALR &  & pcMisaligned6    \\
        (n + 0) & or     & Bool & pcMisalignedBEQ  & pcBEQ            &  & pcMisaligned5    \\
        (n + 0) & or     & Bool & \vdots           & \vdots           &  & \vdots           \\
        (n + 0) & or     & Bool & pcMisalignedBGEU & pcMisaligned1    &  & pcMisaligned     \\
        (n + 0) & bad    &      & pcMisaligned                                              \\
        \hline
        \hline
    \end{tabular}
    \caption[Instruction address misaligned constraint]{Constraining the model on misaligned addresses}\label{fig:badaddress}
\end{figure}

\section{Testing for Correctness}\label{sec:corectness}
To test my model, I compared its results to my RISC-V simulators
(\secref{sec:simulation}) results.

With a given state, both the simulation and the BTOR2 model are run. For both
the iteration maximum is set to 1. The resulting BTOR2 witness can not be
directly compared to the resulting state of the simulation. So I also
implemented a simple converter from witness to state
\cite[src/restate\_witness.c]{repoRV2BTOR}. These two states now can be
compared. I have written a shell script for this at
\cite[sh\_utils/compare\_iterations.sh]{repoRV2BTOR}

To generate RISC-V states, I implemented a fuzzer
\cite[src/state\_fuzzer.c]{repoRV2BTOR} to generate randomised states with one
valid instruction at the address of pc. The fuzzer first chooses an instruction
to test, on this basis it fills all variable parts of the instruction, e.g. rd
or imm. Now all registers relevant to the instruction are filled with a random
64bit value. Also a pc value is generated so that the instruction still fits in
the limited address space of the BTOR2 model. At last if a jump instruction was
chosen, a possible address misalignment is fixed and an address overflow
prevented. The second part is for simplifying later comparison on the resulting
states, as now a correct execution of the instruction always results in the
exact same resulting state although the differences between simulation and the
BTOR2 model.

With this it is possible to start test series. For this I implemented a shell
script, too \cite[sh\_utils/test\_btor2\_model.sh]{repoRV2BTOR}. Also, as with
big amounts of tests, it becomes harder to keep an overview over failed tests.
To counter this I also have written a script to unite all failed tests into one
file and also add some not so easy to access information like instruction name
or immediate value \cite[sh\_utils/diff\_logger.sh]{repoRV2BTOR}.

I have run around 5.000.000 tests on this model without one failing, so I
conclude that my implementation is correct.

\section{Functional vs Relational Next-State Logic}\label{sec:funcVSrel}
F. Schrögendorfer gives two options for the next state logic, a functional
approach \cite[Chapter 6]{bmcOfLockless}, and a relational approach
\cite[Chapter 7]{bmcOfLockless}.