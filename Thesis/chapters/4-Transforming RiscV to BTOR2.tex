\chapter{Transforming RISC-V to BTOR2}\label{chap:riscv_to_btor2}

This chapter addresses the main problem of the thesis: transforming RISC-V code
into the BTOR2 format for benchmarking purposes. My primary reference for this
endeavor is F. Schrögendorfer's master's thesis, \dq Bounded Model Checking in
Lockless Programs\dq \cite{bmcOfLockless}, in which he describes, among other
topics, an encoding concept for a minimal machine in a multiprocessor context
\cite[Chapter 2]{bmcOfLockless} and two approaches to next-state logic: a
functional \cite[Chapter 6]{bmcOfLockless} and a relational \cite[Chapter
    7]{bmcOfLockless} approach. I will focus on the relational approach; a
discussion of both approaches can be found in \secref{sec:funcVSrel}.

\section{The Concept}
To successfully execute a RISC-V instruction, three fundamental steps must
occur in sequence:
\begin{itemize}
    \item Fetch the current instruction from memory
    \item Identify the instruction
    \item Execute the instruction
\end{itemize}
Due to the fixed instruction length of RISC-V, as mentioned in \secref{sec:riscvIsa}, fetching the current instruction is straightforward. Ultimately, we want a node that retrieves a $word$ from memory at the location specified by $pc$.

For basic identification, the $opcode$ must be extracted and checked. Depending
on the opcode, further distinctions between instructions require extracting and
checking $funct3$ and, if necessary, $funct7$. Ultimately, we want a node for
each instruction, which holds a boolean value indicating whether this
instruction was fetched.

To execute the instruction, we need to extract the values of the immediate
$imm$ and, if used, the registers $rs1$ and $rs2$. All instructions only modify
$rd$, $pc$, or memory. Therefore, the next-state logic can be generalized for
these three cases.

Memory is only modified when a store instruction is identified. As all store
instructions share the same type, computing the memory address is consistent
across them. The final step is overwriting the memory at this address.

For the $pc$, except for jump commands, it always increments to point to the
next instruction. The two unconditional jumps, \texttt{JAL} and \texttt{JALR},
must be handled separately. For branch instructions, after determining whether
the relevant condition for the instruction holds, we can generalize, as all
branch instructions execute the same operation from this point onward.

With $rd$, generalization across instructions is not feasible. However, we can
generalize across all possible registers by adding a check in each register's
update function to determine whether the register in question is $rd$.

\section{Encoding}
For better visualisation in the BTOR2 code I will mark all sort-ids in
\textcolor{UniGrey}{grey}, all node-ids in \textcolor{UniRed}{red} and all
non-id numbers \textcolor{UniBlue}{blue}. As described in the BTOR2 syntax
\cite[Figure 1]{btor2}, each line can get an accompanying symbol. Sadly those
cant be used as an alias to the line numbers, but for increased clarity, in the
following figures I will use them as such aliases. With this I can also start
each new figure with the relative line number \texttt{n} and it makes it
feasible to describe processes with algorithms. In the following, i will
describe how I construct a BTOR2 model for a RISC-V state file.

\subsection{Constants}
First off, I added the sorts needed and some general purpose useful constants
into the BTOR2 model as seen in \figref{fig:constants}. Of note is the
Representation of the memory as an array of addressable memory cells of each
1byte. Obviously, the set address space of 16bit is magnitudes away of the
expected address space of 64bit, but representing a 64bit addressable memory
with its resulting $2^{64}B \approx 18 Exabyte$ is not implementable.
Therefore, as I needed a feasible amount of memory space, I artificially chose
a 16bit address space as a soft minimum. With $~65kB$ and therefore programs
with possibly $>10000$ instructions I deemed this memory sufficient for most
use cases. Despite this, the encoding is implemented in such a way that the
address space can be altered with. \todo{benchmark auswirkungen von memory
    size} \begin{figure}
    \centering
    \ttfamily
    \begin{tabular}{>{\color{UniRed}}r l l l l >{\slshape} l}
        \hline
        \hline
        \textcolor{UniGrey}{1} & sort   & bitvec                    & \textcolor{UniBlue}{1}      &                        & Bool          \\
        \textcolor{UniGrey}{2} & sort   & bitvec                    & \textcolor{UniBlue}{16}     &                        & AS            \\
        \textcolor{UniGrey}{3} & sort   & bitvec                    & \textcolor{UniBlue}{8}      &                        & B             \\
        \textcolor{UniGrey}{4} & sort   & bitvec                    & \textcolor{UniBlue}{16}     &                        & H             \\
        \textcolor{UniGrey}{5} & sort   & bitvec                    & \textcolor{UniBlue}{32}     &                        & W             \\
        \textcolor{UniGrey}{6} & sort   & bitvec                    & \textcolor{UniBlue}{64}     &                        & D             \\
        \textcolor{UniGrey}{7} & sort   & array                     & \textcolor{UniGrey}{2}      & \textcolor{UniGrey}{3} & Mem           \\
        \\
        8                      & one    & \textcolor{UniGrey}{Bool} &                             &                        & true          \\
        9                      & zero   & \textcolor{UniGrey}{Bool} &                             &                        & false         \\
        10                     & one    & \textcolor{UniGrey}{AS}   &                             &                        & addressInc    \\
        11                     & constd & \textcolor{UniGrey}{AS}   & \textcolor{UniBlue}{4}      &                        & pcInc         \\
        12                     & zero   & \textcolor{UniGrey}{B}    &                             &                        & emptyCell     \\
        13                     & one    & \textcolor{UniGrey}{W}    &                             &                        & bitPicker     \\
        14                     & zero   & \textcolor{UniGrey}{D}    &                             &                        & emptyReg      \\
        \\
        15                     & consth & \textcolor{UniGrey}{W}    & \textcolor{UniBlue}{01F}    &                        & 5Bitmask      \\
        16                     & consth & \textcolor{UniGrey}{W}    & \textcolor{UniBlue}{03F}    &                        & 6Bitmask      \\
        17                     & consth & \textcolor{UniGrey}{W}    & \textcolor{UniBlue}{07F}    &                        & 7Bitmask      \\
        18                     & consth & \textcolor{UniGrey}{W}    & \textcolor{UniBlue}{0FFF}   &                        & 12Bitmask     \\
        19                     & consth & \textcolor{UniGrey}{W}    & \textcolor{UniBlue}{0FFFFF} &                        & 20Bitmask     \\
        \\
        20                     & constd & \textcolor{UniGrey}{W}    & \textcolor{UniBlue}{7}      &                        & shiftToRd     \\
        21                     & constd & \textcolor{UniGrey}{W}    & \textcolor{UniBlue}{15}     &                        & shiftToRs1    \\
        22                     & constd & \textcolor{UniGrey}{W}    & \textcolor{UniBlue}{20}     &                        & shiftToRs2    \\
        23                     & constd & \textcolor{UniGrey}{W}    & \textcolor{UniBlue}{12}     &                        & shiftToFunct3 \\
        24                     & constd & \textcolor{UniGrey}{W}    & \textcolor{UniBlue}{25}     &                        & shiftToFunct7 \\
        25                     & constd & \textcolor{UniGrey}{W}    & \textcolor{UniBlue}{5}      &                        & shiftBy5      \\
        26                     & constd & \textcolor{UniGrey}{W}    & \textcolor{UniBlue}{11}     &                        & shiftBy11     \\
        \\
        27                     & constd & \textcolor{UniGrey}{W}    & \textcolor{UniBlue}{3}      &                        & load          \\
        28                     & constd & \textcolor{UniGrey}{W}    & \textcolor{UniBlue}{19}     &                        & opImm         \\
        29                     & constd & \textcolor{UniGrey}{W}    & \textcolor{UniBlue}{23}     &                        & auipc         \\
        30                     & constd & \textcolor{UniGrey}{W}    & \textcolor{UniBlue}{27}     &                        & opImm32       \\
        31                     & constd & \textcolor{UniGrey}{W}    & \textcolor{UniBlue}{35}     &                        & store         \\
        32                     & constd & \textcolor{UniGrey}{W}    & \textcolor{UniBlue}{51}     &                        & op            \\
        33                     & constd & \textcolor{UniGrey}{W}    & \textcolor{UniBlue}{55}     &                        & lui           \\
        34                     & constd & \textcolor{UniGrey}{W}    & \textcolor{UniBlue}{59}     &                        & op32          \\
        35                     & constd & \textcolor{UniGrey}{W}    & \textcolor{UniBlue}{99}     &                        & branch        \\
        36                     & constd & \textcolor{UniGrey}{W}    & \textcolor{UniBlue}{103}    &                        & jalr          \\
        37                     & constd & \textcolor{UniGrey}{W}    & \textcolor{UniBlue}{111}    &                        & jal           \\
        \hline
        \hline
    \end{tabular}
    \caption[Sorts and non-progressive Constants]{Sorts and non-progressive
        Constants for encoding RISC-V in BTOR2}\label{fig:constants}
\end{figure}



\subsection{State Representation}
The next logical step is defining a representation of a RISC-V state. Tis is
straightforward as shown in \figref{fig:states}. I also introduced a flag for
each register in my code. They track if the register was written to and makes
it possible to shorten a state file transformed from a witness to only the
relevant registers. As they have no impact on the operation of the BTOR2 model,
I will not mention them again. \begin{figure}
    %\centering
    \begin{center}

        \ttfamily
        %\begin{minipage}{.5\linewidth}
        \begin{tabular}[t]{>{\color{UniRed}}r l >{\color{UniGrey}}l l >{\slshape} l}
            \hline
            \hline
            (n + 0)  & state & D &  & x0  \\
            (n + 1)  & state & D &  & x1  \\
            (n + 2)  & state & D &  & x2  \\
            (n + 3)  & state & D &  & x3  \\
            (n + 4)  & state & D &  & x4  \\
            (n + 5)  & state & D &  & x5  \\
            (n + 6)  & state & D &  & x6  \\
            (n + 7)  & state & D &  & x7  \\
            (n + 8)  & state & D &  & x8  \\
            (n + 9)  & state & D &  & x9  \\
            (n + 10) & state & D &  & x10 \\
            (n + 11) & state & D &  & x11 \\
            (n + 12) & state & D &  & x12 \\
            (n + 13) & state & D &  & x13 \\
            (n + 14) & state & D &  & x14 \\
            (n + 15) & state & D &  & x15 \\
            (n + 16) & state & D &  & x16 \\
        \end{tabular}
        \hspace{1cm}
        %\end{minipage}%
        %\begin{minipage}{.5\linewidth}
        \begin{tabular}[t]{>{\color{UniRed}}r l >{\color{UniGrey}}l l >{\slshape} l}
            \\
            (n + 17) & state & D   &  & x17    \\
            (n + 18) & state & D   &  & x18    \\
            (n + 19) & state & D   &  & x19    \\
            (n + 20) & state & D   &  & x20    \\
            (n + 21) & state & D   &  & x21    \\
            (n + 22) & state & D   &  & x22    \\
            (n + 23) & state & D   &  & x23    \\
            (n + 24) & state & D   &  & x24    \\
            (n + 25) & state & D   &  & x25    \\
            (n + 26) & state & D   &  & x26    \\
            (n + 27) & state & D   &  & x27    \\
            (n + 28) & state & D   &  & x28    \\
            (n + 29) & state & D   &  & x29    \\
            (n + 30) & state & D   &  & x30    \\
            (n + 31) & state & D   &  & x31    \\
            (n + 32) & state & AS  &  & pc     \\
            (n + 33) & state & Mem &  & memory \\
            \hline
            \hline
        \end{tabular}
        %\end{minipage}
    \end{center}
    \caption[State representation for transforming RISC-V to BTOR2]{State representation for encoding}\label{fig:states}
\end{figure}



\subsection{Initialization}
To initialize a state in BTOR2 from a RISC-V state file, the values in the
registers must be loaded as constants, and for each memory address mentioned in
the state file, the value and address has to be loaded as constants. Due to the
inability to represent a full 64bit address space, the shrinking of the address
space from state file to BTOR2 model must be handled. I decided to just
initialiase the addresses up to the BTOR2 model address space maximum and cut
all others in the state file as I deem this the most predictable behaivour.
Everything not mentioned in the state file will be zero-initialised. At last
these constants must be used to initialise the state. For the registers this is
straight forward, for the memory we must first write all memory adresses into a
placeholder array wich then we can use to initialise the real memory. Due to
constraints in BTOR2, these constants have to be defined \textbf{before} the
states, but initialisation with the values must happen after the states. This
means that this initialisation process \textbf{wrappes around} the state
representation. The generation of constants is shown in
\algoref{alg:generateconstantsfromstate}, whereas the actual initialization is
shown in \algoref{alg:initstate}.
\begin{algorithm}
    $truePc \leftarrow$ value of pc in state file\\
    $maxPc \leftarrow$ number of addresses in BTOR2 model\\
    \textcolor{UniBlue}{pcValue} $\leftarrow truePc$ modulo $maxPc$\\
    add to model:\\
    \begin{tabular}[h]{>{\color{UniRed}}r l >{\color{UniGrey}}l l >{\slshape} l}
        \hline
        \hline
        \ttfamily
        (n + 0) & constd & AS & \textcolor{UniBlue}{\rmfamily\textsl{pcValue}} & pcConst \\
        \hline
        \hline
    \end{tabular}\\
    \textcolor{UniRed}{n} += 1
    \BlankLine

    \For{every register \textcolor{Green}{$x_i$}}{
        \If{register is initialised in state file}{
            \textcolor{UniBlue}{$registerValue$} $\leftarrow$ value of \textcolor{Green}{$x_i$}\\
            \If{\textcolor{UniBlue}{$registerValue$} $\neq 0$}{
                add to model:\\
                \begin{tabular}[h]{>{\color{UniRed}}r l >{\color{UniGrey}}l l >{\slshape} l}
                    \hline
                    \hline
                    \ttfamily
                    (n + 0) & constd & D & \textcolor{UniBlue}{\rmfamily$registerValue$} & \textcolor{Green}{$x_i$}Const \\
                    \hline
                    \hline
                \end{tabular}\\
                \textcolor{UniRed}{n} += 1
            }
        }
    }
    \BlankLine
    add to model:\\

    \begin{tabular}[h]{>{\ttfamily\color{UniRed}}r >{\ttfamily}l >{\ttfamily\color{UniGrey}}l >{\ttfamily}l >{\slshape} l}
        \hline
        \hline
        (n + 0) & state & Mem &                                     & memPH \\
        (n + 1) & init  & Mem & \textcolor{UniRed}{$memPH$ (n + 0)} &       \\
        \hline
        \hline
    \end{tabular}\\
    \textcolor{UniRed}{n} += 2\\

    $\textcolor{UniRed}{lastPH} \leftarrow \textcolor{UniRed}{memPH}$\\
    $allInitialCells \leftarrow$ all initialised memory cells in the state file\\
    $cutInitialCells \leftarrow$ remove all cells with address over maxPc\\
    \For{every cell \textcolor{Green}{$c$} in $cutInitialCells$}{
        $address \leftarrow$ address of \textcolor{Green}{$c$}\\
        $value \leftarrow$ value of \textcolor{Green}{$c$}\\
        add to model:\\
        \begin{tabular}[h]{>{\ttfamily\color{UniRed}}r >{\ttfamily}l >{\ttfamily\color{UniGrey}}l >{\ttfamily}l >{\slshape} l}
            \hline
            \hline
            \ttfamily
            (n + 0) & constd & AS  & \textcolor{UniBlue}{\rmfamily$address$}               &                             \\
            (n + 1) & constd & B   & \textcolor{UniBlue}{\rmfamily$value$}                 &                             \\
            (n + 2) & write  & Mem & \textcolor{UniRed}{\textrm{$lastPH$} (n + 0) (n + 1)} & PHAfter\textcolor{Green}{C} \\
            \hline
            \hline
        \end{tabular}\\
        \textcolor{UniRed}{n} += 3\\
        $\textcolor{UniRed}{lastPH} \leftarrow \textcolor{UniRed}{PHAfter\textcolor{Green}{C}}$\\
    }
    keep  $\textcolor{UniRed}{lastPH}$ for initialisation

    \caption[Generating initialisation constants]{Generating initialisation constants from state file in BTOR2}\label{alg:generateconstantsfromstate}
\end{algorithm}
\begin{algorithm}
    add to model:\\
    \begin{tabular}[h]{>{\ttfamily\color{UniRed}}r >{\ttfamily}l >{\ttfamily\color{UniGrey}}l >{\slshape\color{UniRed}}l >{\slshape} l}
        \hline
        \hline
        \ttfamily
        n & init & AS & pc\ \ pcConst & \\
        \hline
        \hline
    \end{tabular}\\
    \BlankLine
    \For{every register \textcolor{Green}{$x_i$}}{
        \If{\textcolor{Green}{$x_i$}\textcolor{UniRed}{Const} was defined}{
            add to model:\\
            \begin{tabular}[h]{>{\ttfamily\color{UniRed}}r >{\ttfamily}l >{\ttfamily\color{UniGrey}}l >{\slshape\color{UniRed}}l >{\slshape} l}
                \hline
                \hline
                \ttfamily
                n & init & D & \textcolor{Green}{$x_i$}\ \ \textcolor{Green}{$x_i$}Const & \\
                \hline
                \hline
            \end{tabular}\\
        }
    }
    \BlankLine
    With \textcolor{Green}{$lastPH$} from \algoref{alg:generateconstantsfromstate}\\
    add to model:\\
    \begin{tabular}[h]{>{\ttfamily\color{UniRed}}r >{\ttfamily}l >{\ttfamily\color{UniGrey}}l >{\slshape\color{UniRed}}l >{\slshape} l}
        \hline
        \hline
        \ttfamily
        n & init & Mem & memory\ \ \color{Green}lastPh & \\
        \hline
        \hline
    \end{tabular}\\
    \caption[Initialising states]{Initialising states in the BTOR2 model}\label{alg:initstate}
\end{algorithm}

\subsection{Fetching the current instruction}
To fetch the current instruction, i read the 4 bytes of the instruction and
concatenate them as seen in \figref{fig:fetching}
\begin{figure}
    \centering
    \begin{tabular}[h]{>{\ttfamily\color{UniRed}}r >{\ttfamily}l >{\ttfamily\color{UniGrey}}l >{\slshape\color{UniRed}}l >{\ttfamily\color{UniRed}}l >{\slshape} l}
        \hline
        \hline
        \ttfamily
        (n + 0) & add    & AS & addressInc               & \rmfamily\slshape pc &       \\
        (n + 1) & add    & AS & addressInc               & (n + 0)              &       \\
        (n + 2) & add    & AS & addressInc               & (n + 1)              &       \\\\
        (n + 3) & read   & B  & memory                   & \rmfamily\slshape pc &       \\
        (n + 4) & read   & B  & memory                   & (n + 0)              &       \\
        (n + 5) & read   & B  & memory                   & (n + 1)              &       \\
        (n + 6) & read   & B  & memory                   & (n + 2)              &       \\\\
        (n + 7) & concat & H  & \upshape\ttfamily(n + 4) & (n + 3)              &       \\
        (n + 8) & concat & H  & \upshape\ttfamily(n + 6) & (n + 5)              &       \\
        (n + 9) & concat & W  & \upshape\ttfamily(n + 8) & (n + 7)              & instr \\
        \hline
        \hline
    \end{tabular}
    \caption[Fetching instruction]{Fetching the current instruction from memory}\label{fig:fetching}
\end{figure}

\subsection{Deconstruction of the instruction}
Now having the instruction, we can deconstruct it to extract the $opcode$,
$rd$, $rs1$, $rs2$, $funct3$, $funct7$ and $imm$. For everything apart of
$imm$, this can be done by a shift and a masking. This is shown in
\figref{fig:extractNOimm}. The immediate on the other hand must be first
constructed from its subfields, which can be referenced in
\figref{fig:rv64i_formats}. In the BTOR2 model this looks like in
\figref{fig:extractimmbytype}. \todo{Reference to same method in riscvsim}
There are three things i want to point out:\\ First, some of the immediate
subfields overlap exactly. I made use of this fact in lines (n + 1) with the
overlap of $imm[11:5]$ of I- and S-type, and (n + 21) with J- and B-types
$imm[10:5]$ overlap. Second, as described in \secref{sec:riscvIsa} the
immediate is always sign-extended. To archive this we make arithmetic right
shifts, which do sign extension for us and with this pull our highest immediate
bit to its correct place. Third, at line (n + 8), for sign extension we must
shift right by 19. As this matches the opcode for arithmetic instructions with
immediates, so I used this and did not create a new constant.

Now I have \textsl{iTypeImm}, \textsl{sTypeImm}, \textsl{bTypeImm},
\textsl{uTypeImm} and \textsl{jTypeImm}. But it would be easier to just have
one node \textsl{imm} where we can reference the immediate value regardless of
the instruction. This is done in \figref{fig:findingImm}, where first I defined
Bools wich check all opcodes that are neither R-type nor I-type. Then I chained
if-then-else nodes to catch instructions that are of J-type, U-Type, B-Type or
S-type. If the instruction is none of them, I can safely default to I-type as
R-type does not handle with an immediate value.

\begin{figure}
    \centering
    \begin{tabular}[h]{>{\ttfamily\color{UniRed}}r >{\ttfamily}l >{\ttfamily\color{UniGrey}}l >{\slshape\color{UniRed}}l >{\slshape\color{UniRed}}l >{\slshape} l}
        \hline
        \hline
        (n + 0) & and & W & instr                    & 7Bitmask      & opcode \\
        (n + 1) & srl & W & instr                    & shiftToRd     &        \\
        (n + 2) & and & W & \upshape\ttfamily(n + 1) & 5Bitmask      & rd     \\
        (n + 3) & srl & W & instr                    & shiftToRs1    &        \\
        (n + 4) & and & W & \upshape\ttfamily(n + 3) & 5Bitmask      & rs1    \\
        (n + 5) & srl & W & instr                    & shiftToRs2    &        \\
        (n + 6) & and & W & \upshape\ttfamily(n + 5) & 5Bitmask      & rs2    \\
        (n + 7) & srl & W & instr                    & shiftToFunct3 &        \\
        (n + 8) & and & W & \upshape\ttfamily(n + 7) & shiftRd       & funct3 \\
        (n + 9) & srl & W & instr                    & shiftToFunct7 & funct7 \\
        \hline
        \hline
    \end{tabular}
    \caption[Extraction (without immediate)]{Extraction of values from the instruction without $imm$}\label{fig:extractNOimm}
\end{figure}
\begin{figure}
    \centering
    \begin{tabular}[h]{>{\ttfamily\color{UniRed}}r >{\ttfamily}l >{\ttfamily\color{UniGrey}}l >{\slshape\color{UniRed}}l >{\slshape\color{UniRed}}l >{\slshape} l}
        \hline
        \hline
        (n + 0)  & sra & W & instr                     & shiftToRs2                                     & iTypeImm \\
        \\
        (n + 1)  & and & W & iTypeImm                  & \textcolor{Black}{\upshape\ttfamily-}5Bitmask  & s[11:5]  \\
        (n + 2)  & add & W & s[11:5]                   & rd                                             & sTypeImm \\
        \\
        (n + 3)  & and & W & rd                        & \textcolor{Black}{\upshape\ttfamily-}bitPicker & b[4:0]   \\
        (n + 4)  & and & W & funct7                    & 6Bitmask                                       &          \\
        (n + 5)  & sll & W & \upshape\ttfamily(n + 4)  & shiftBy5                                       & b[10:5]  \\
        (n + 6)  & and & W & bitPicker                 & rd                                             &          \\
        (n + 7)  & sll & W & \upshape\ttfamily(n + 6)  & shiftBy11                                      & b[11]    \\
        (n + 8)  & sra & W & instr                     & mathI                                          &          \\
        (n + 9)  & and & W & \upshape\ttfamily(n + 8)  & 12Bitmask                                      & b[31:12] \\
        (n + 10) & add & W & b[10:5]                   & b[4:0]                                         &          \\
        (n + 11) & add & W & b[11]                     & \upshape\ttfamily(n + 10)                      &          \\
        (n + 12) & add & W & b[31:12]                  & \upshape\ttfamily(n + 11)                      & bTypeImm \\
        \\
        (n + 13) & and & W & instr                     & \textcolor{Black}{\upshape\ttfamily-}12Bitmask & uTypeImm \\
        \\
        (n + 14) & and & W & rs2                       & \textcolor{Black}{\upshape\ttfamily-}bitPicker & j[4:0]   \\
        (n + 15) & and & W & rs2                       & bitPicker                                      &          \\
        (n + 16) & sll & W & \upshape\ttfamily(n + 15) & shiftBy11                                      & j[11]    \\
        (n + 17) & sll & W & funct3                    & shiftToFunct3                                  & j[14:12] \\
        (n + 18) & sll & W & rs1                       & shiftToRs1                                     & j[19:15] \\
        (n + 19) & sra & W & instr                     & shiftBy11                                      &          \\
        (n + 20) & and & W & \upshape\ttfamily(n + 19) & \textcolor{Black}{\upshape\ttfamily-}20Bitmask & j[31:20] \\
        (n + 21) & add & W & b[10:5]                   & j[4:0]                                         &          \\
        (n + 22) & add & W & j[11]                     & \upshape\ttfamily(n + 21)                      &          \\
        (n + 23) & add & W & j[14:12]                  & \upshape\ttfamily(n + 22)                      &          \\
        (n + 24) & add & W & j[19:15]                  & \upshape\ttfamily(n + 23)                      &          \\
        (n + 25) & add & W & j[31:20]                  & \upshape\ttfamily(n + 24)                      & jTypeImm \\
        \hline
        \hline
    \end{tabular}
    \caption[Extraction of the different $imm$ Types]{Extraction of all $imm$ Types from the Instruction}\label{fig:extractimmbytype}
\end{figure}

% ; Basics
% 1 sort bitvec 1 Bool
% 2 sort bitvec %d AS
% 3 sort bitvec 8 B
% 4 sort bitvec 16 H
% 5 sort bitvec 32 W
% 6 sort bitvec 64 D
% 7 sort array 2 3 Mem
% 8 zero 5 empty_reg
% 9 constd 5 31 register_bitmask
% 10 constd 5 7 shift_rd
% 11 constd 5 15 shift_rs1
% 12 constd 5 20 shift_rs2
% 13 constd 5 12 shift_funct3
% 14 constd 5 25 shift_funct7
% 15 one 5 bit_picker
% 16 one 1 true
% 17 zero 1 false
\begin{figure}
    \centering
    \begin{tabular}[h]{>{\ttfamily\color{UniRed}}r >{\ttfamily}l >{\ttfamily\color{UniGrey}}l >{\slshape\color{UniRed}}l >{\slshape\color{UniRed}}l >{\slshape\color{UniRed}}l >{\slshape} l}
        \hline
        \hline
        (n + 0) & eq  & Bool & opcode  & store    &          & isSType \\
        (n + 1) & eq  & Bool & opcode  & branch   &          & isBType \\
        (n + 2) & eq  & Bool & opcode  & auipc    &          & uType1  \\
        (n + 3) & eq  & Bool & opcode  & lui      &          & uType2  \\
        (n + 4) & or  & Bool & uType1  & uType2   &          & isUType \\
        (n + 5) & eq  & Bool & opcode  & jump     &          & isJType \\
        \\
        (n + 6) & ite & W    & isSType & sTypeImm & iTypeImm & checkS  \\
        (n + 7) & ite & W    & isBType & bTypeImm & checkS   & checkB  \\
        (n + 8) & ite & W    & isUType & uTypeImm & checkB   & checkU  \\
        (n + 9) & ite & W    & isJType & jTypeImm & checkU   & imm     \\
        \hline
        \hline
    \end{tabular}
    \caption[Finding the correct immediate]{Finding the correct immediate by opcode}\label{fig:findingImm}
\end{figure}

\subsection{Instruction Detection}

For the next-state logic, the only thing left that we need to know is the
actual current command. So I defined a check
\textsl{is\textcolor{Green}{Instruction}} for each instruction. As this is
quite repetetive, \algoref{alg:commanddetection} describes a generalised
approach to reach these Bools. An example for each instruction subgroup in
\algoref{alg:commanddetection} can be found in \figref{fig:detectionexample}.

\begin{algorithm}
    add to model:\\
    \begin{tabular}[h]{>{\ttfamily\color{UniRed}}r >{\ttfamily}l >{\ttfamily\color{UniGrey}}l >{\slshape\color{UniRed}}l >{\slshape\color{UniRed}}l >{\slshape\color{UniRed}}l >{\slshape} l}
        \hline
        \hline
        (n + 0)  & eq & Bool & opcode & load    &  & isLoad    \\
        (n + 1)  & eq & Bool & opcode & opImm   &  & isOpImm   \\
        (n + 2)  & eq & Bool & opcode & auipc   &  & isAUIPC   \\
        (n + 3)  & eq & Bool & opcode & opImm32 &  & isOpImm32 \\
        (n + 4)  & eq & Bool & opcode & store   &  & isStore   \\
        (n + 5)  & eq & Bool & opcode & op      &  & isOp      \\
        (n + 6)  & eq & Bool & opcode & lui     &  & isLUI     \\
        (n + 7)  & eq & Bool & opcode & op32    &  & isOp32    \\
        (n + 8)  & eq & Bool & opcode & branch  &  & isBranch  \\
        (n + 9)  & eq & Bool & opcode & jalr    &  & isJALR    \\
        (n + 10) & eq & Bool & opcode & jal     &  & isJAL     \\
        \hline
        \hline
    \end{tabular}\\
    \textcolor{UniRed}{n} += 11\\
    \For{\textcolor{Green}{i} from 0 to 7}{
        add to model:\\
        \begin{tabular}[h]{>{\ttfamily\color{UniRed}}r >{\ttfamily}l >{\ttfamily\color{UniGrey}}l >{\slshape\color{UniRed}}l >{\slshape\color{UniRed}}l >{\slshape\color{UniRed}}l >{\slshape} l}
            \hline
            \hline
            (n + 0) & constd & W    & \texttt{\upshape\color{Green}i} &                           &  & \textcolor{Green}{i}Const    \\
            (n + 1) & eq     & Bool & funct3                          & \textcolor{Green}{i}Const &  & is\textcolor{Green}{i}Funct3 \\
            \hline
            \hline
        \end{tabular}\\
        \textcolor{UniRed}{n} += 2\\
    }
    $onlyOp \leftarrow$ [LUI, AUIPC, JAL, JALR]\\
    $needsf7 \leftarrow$ [SRL, SRA, SRLI, SRAI, SRLW, SRAW, SRLWI, SRAWI, ADD, SUB, ADDW, SUBW]\\
    $rest \leftarrow$[ all other instructions ]\\
    \BlankLine
    \For{ all instructions \textcolor{Green}{I} in $onlyOp$}{
        $is\textcolor{Green}{I}$ is already defined
    }
    \For{all instructions  \textcolor{Green}{I} in $rest$ }{
        $opname \leftarrow$ opcode name of \textcolor{Green}{I}\\
        $f3val \leftarrow$ expected funct3 of \textcolor{Green}{I} as digit\\
        add to model:\\
        \begin{tabular}[h]{>{\ttfamily\color{UniRed}}r >{\ttfamily}l >{\ttfamily\color{UniGrey}}l >{\slshape\color{UniRed}}l >{\slshape\color{UniRed}}l >{\slshape\color{UniRed}}l >{\slshape} l}
            \hline
            \hline
            n & and & Bool & is\textcolor{Black}{opname} & is\textcolor{Black}{f3val}Funct3 &  & is\textcolor{Green}{I} \\
            \hline
            \hline
        \end{tabular}\\
        \textcolor{UniRed}{n} += 1\\
    }
    \For{all instructions  \textcolor{Green}{I} in $needsf7$ }{
        $opname \leftarrow$ opcode name of \textcolor{Green}{I}\\
        $f3val \leftarrow$ expected funct3 of \textcolor{Green}{I} as digit\\
        $f7hex \leftarrow$ expected funct7 of \textcolor{Green}{I} as hexadecimal number\\
        add to model:\\
        \begin{tabular}[h]{>{\ttfamily\color{UniRed}}r >{\ttfamily}l >{\ttfamily\color{UniGrey}}l >{\slshape\color{UniRed}}l >{\slshape\color{UniRed}}l >{\slshape\color{UniRed}}l >{\slshape} l}
            \hline
            \hline
            (n + 0) & consth & W    & \textcolor{UniBlue}{f7hex}       &                            &  & \textcolor{Green}{I}f7     \\
            (n + 1) & eq     & Bool & funct7                           & \textcolor{Green}{I}f7     &  & fitsF7\textcolor{Green}{I} \\
            (n + 2) & and    & Bool & is\textcolor{Black}{f3val}Funct3 & fitsF7\textcolor{Green}{I} &  & fitsF3\textcolor{Green}{I} \\
            (n + 3) & and    & Bool & is\textcolor{Black}{opname}      & fitsF3\textcolor{Green}{I} &  & is\textcolor{Green}{I}     \\
            \hline
            \hline
        \end{tabular}\\
        \textcolor{UniRed}{n} += 4\\
    }
    \caption[Instruction detection]{Generalised approach to instruction detection}\label{alg:commanddetection}
\end{algorithm}
\begin{figure}
    \centering
    \begin{tabular}[h]{>{\ttfamily\color{UniRed}}r >{\ttfamily}l >{\ttfamily\color{UniGrey}}l >{\slshape\color{UniRed}}l >{\slshape\color{UniRed}}l >{\slshape\color{UniRed}}l >{\slshape} l}
        \hline
        \hline
        \multicolumn{7}{l}{\rmfamily(\slshape isJALR \upshape already exists)}                   \\
        \hline
        n\ \    & and    & Bool & isLoad                  & is5Funct3                &  & isLHU  \\
        \hline
        (n + 0) & consth & W    & \textcolor{UniBlue}{20} &                          &  &        \\
        (n + 1) & eq     & Bool & funct7                  & \upshape\ttfamily(n + 0) &  &        \\
        (n + 2) & and    & Bool & is0Funct3               & \upshape\ttfamily(n + 1) &  &        \\
        (n + 3) & and    & Bool & isLoad                  & \upshape\ttfamily(n + 2) &  & isSUBW \\
        \hline
        \hline
    \end{tabular}
    \caption[Examples for Instruction Detection]{Instruction Detection of \texttt{JALR}, \texttt{LHU} and \texttt{SUBW} as described in \algoref{alg:commanddetection}}\label{fig:detectionexample}
\end{figure}

\subsection{Next-State Logic}
\subsubsection{$rd$}
\subsubsection{$pc$}
\subsubsection{Memory}
\subsection{Constraints}
\todo{Iterations counter auch hier}
\section{Testing for Correctness}\label{sec:corectness}
\subsection{State Fuzzer}
\subsection{Automated Logging}

\section{Functional vs Relational Next-State Logic}\label{sec:funcVSrel}