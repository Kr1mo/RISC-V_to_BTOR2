\chapter{Transforming RISC-V to BTOR2}\label{chap:riscv_to_btor2}
Now to the main problem of the thesis: Transforming RISC-V code into the BTOR2
format for the benchmarks. My main source in this endeavour is F.
Schrögendorfers master thesis \dq Bounded Model Checking in Lockless
Programs\dq \cite{bmcOfLockless} where he, among other things, describes an
encoding concept for a minimal machine in a multiprocessor context
\cite[chapter 2]{bmcOfLockless} and two approaches to a Next-State-Logic, a
funtional \cite[Chapter 6]{bmcOfLockless} and a relational \cite[Chapter
    7]{bmcOfLockless} I will focus on the relational approach, an argument on both
approaches is found in \secref{sec:funcVSrel}

\section{The Concept}
To successfully execute RISC-V instruction, three basic steps have to happen in
order:
\begin{itemize}
    \item Fetching the current instruction from memory
    \item Identifying this instruction
    \item Execute this instruction
\end{itemize}
Because of the fixed instruction length of RISC-V, earlier mentioned in \secref{sec:riscvIsa}, fetching the current instruction becomes trivially easy.

For Identifying,

\section{Encoding}
\subsection{Constants}
\subsection{State Representation}
\subsection{Initialization}
\subsection{Fetching}
\subsection{Computing values}
\subsubsection{Opcode}
\subsubsection{funct3 \& funct7}
\subsubsection{Registers}
\subsubsection{Immediate}
\subsection{Command Detection}
\subsection{Next-State-Logic}
\subsection{Constraints}
\section{Testing for Correctness}
\subsection{State Fuzzer}
\subsection{Automated Logging}
\section{Functional vs Relational Next-State-Logic}\label{sec:funcVSrel}