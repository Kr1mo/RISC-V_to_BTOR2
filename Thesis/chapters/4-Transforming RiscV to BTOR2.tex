\chapter{Transforming RISC-V to BTOR2}\label{chap:riscv_to_btor2}

This chapter addresses the main problem of the thesis: transforming RISC-V code
into the BTOR2 format for benchmarking purposes. My primary reference for this
endeavor is F. Schrögendorfer's master's thesis, \dq Bounded Model Checking in
Lockless Programs\dq \cite{bmcOfLockless}, in which he describes, among other
topics, an encoding concept for a minimal machine in a multiprocessor context
\cite[Chapter 2]{bmcOfLockless} and two approaches to next-state logic: a
functional \cite[Chapter 6]{bmcOfLockless} and a relational \cite[Chapter
    7]{bmcOfLockless} approach. I will focus on the relational approach; a
discussion of both approaches can be found in \secref{sec:funcVSrel}.

\section{The Concept}
To successfully execute a RISC-V instruction, three fundamental steps must
occur in sequence:
\begin{itemize}
    \item Fetch the current instruction from memory
    \item Identify the instruction
    \item Execute the instruction
\end{itemize}
Due to the fixed instruction length of RISC-V, as mentioned in \secref{sec:riscvIsa}, fetching the current instruction is straightforward. Ultimately, we want a node that retrieves a $word$ from memory at the location specified by $pc$.

For basic identification, the $opcode$ must be extracted and checked. Depending
on the opcode, further distinctions between instructions require extracting and
checking $funct3$ and, if necessary, $funct7$. Ultimately, we want a node for
each instruction, which holds a boolean value indicating whether this
instruction was fetched.

To execute the instruction, we need to extract the values of the immediate
$imm$ and, if used, the registers $rs1$ and $rs2$. All instructions only modify
$rd$, $pc$, or memory. Therefore, the next-state logic can be generalized for
these three cases.

Memory is only modified when a store instruction is identified. As all store
instructions share the same type, computing the memory address is consistent
across them. The final step is overwriting the memory at this address.

For the $pc$, except for jump commands, it always increments to point to the
next instruction. The two unconditional jumps, \texttt{JAL} and \texttt{JALR},
must be handled separately. For branch instructions, after determining whether
the relevant condition for the instruction holds, we can generalize, as all
branch instructions execute the same operation from this point onward.

With $rd$, generalization across instructions is not feasible. However, we can
generalize across all possible registers by adding a check in each register's
update function to determine whether the register in question is $rd$.

\section{Encoding}
For better visualisation in the BTOR2 code I will mark all sort-ids in
\textcolor{UniGrey}{grey}, all node-ids in \textcolor{UniRed}{red} and all
non-id numbers \textcolor{UniBlue}{blue}.
As described in the BTOR2 syntax \cite[Figure 1]{btor2}, each line can get an accompanying symbol. Sadly those cant be used as an alias to the line numbers, but for increased clarity, in the following figures I will use them as such aliases.


\subsection{Constants}
First off, I added the sorts needed and some general purpose useful constants
into the BTOR2 model as seen in \figref{fig:constants}.
\begin{figure}[h]
    \centering
    \ttfamily
    \begin{tabular}{>{\color{UniRed}}r l l l >{\itshape} l}
        \textcolor{UniGrey}{1} & sort   & bitvec                 & \textcolor{UniBlue}{1}   & Boolean           \\
        \textcolor{UniGrey}{2} & sort   & bitvec                 & \textcolor{UniBlue}{16}  & Address\_space    \\
        \textcolor{UniGrey}{3} & sort   & bitvec                 & \textcolor{UniBlue}{8}   & Memory\_cell      \\
        \textcolor{UniGrey}{4} & sort   & bitvec                 & \textcolor{UniBlue}{32}  & Command           \\
        \textcolor{UniGrey}{5} & sort   & bitvec                 & \textcolor{UniBlue}{64}  & Register          \\
        \textcolor{UniGrey}{6} & sort   & array                  & \textcolor{UniGrey}{2 3} & Memory            \\
        7                      & zero   & \textcolor{UniGrey}{5} &                          & empty\_reg        \\
        8                      & constd & \textcolor{UniGrey}{4} & \textcolor{UniBlue}{31}  & register\_bitmask \\
        9                      & constd & \textcolor{UniGrey}{4} & \textcolor{UniBlue}{7}   & shift\_rd         \\
        10                     & constd & \textcolor{UniGrey}{4} & \textcolor{UniBlue}{15}  & shift\_rs1        \\
        11                     & constd & \textcolor{UniGrey}{4} & \textcolor{UniBlue}{20}  & shift\_rs2        \\
        12                     & constd & \textcolor{UniGrey}{4} & \textcolor{UniBlue}{12}  & shift\_funct3     \\
        13                     & constd & \textcolor{UniGrey}{4} & \textcolor{UniBlue}{25}  & shift\_funct7     \\
        14                     & one    & \textcolor{UniGrey}{4} &                          & bit\_picker       \\
        15                     & one    & \textcolor{UniGrey}{1} &                          & true              \\
        16                     & zero   & \textcolor{UniGrey}{1} &                          & false             \\
    \end{tabular}
    \caption[Constants for transforming RISC-V to BTOR2]{Constants for encoding}\label{fig:constants}
\end{figure}

 Of note is the Representation of
the memory as an array of addressable memory cells of each 1byte. Obviously,
the set address space of 16bit is magnitudes away of the expected address space
of 64bit, but representing a 64bit addressable memory with its resulting
$2^{64}B \approx 18 Exabyte$ is not implementable. Therefore, as I needed a
feasible amount of memory space, I artificially chose a 16bit address space as
a soft minimum. With $~65kB$ and therefore programs with possibly $>10000$
instructions I deemed this memory sufficient for most use cases. Despite this,
the encoding is implemented in such a way that the address space can be altered
with. \todo{benchmark auswirkungen von memory size}

\subsection{State Representation}

The next logical step is defining a representation of a RISC-V state. Tis is straightforward as shown in \figref{fig:states}
\begin{figure}
    %\centering
    \begin{center}

        \ttfamily
        %\begin{minipage}{.5\linewidth}
        \begin{tabular}[t]{>{\color{UniRed}}r l >{\color{UniGrey}}l l >{\slshape} l}
            \hline
            \hline
            (n + 0)  & state & D &  & x0  \\
            (n + 1)  & state & D &  & x1  \\
            (n + 2)  & state & D &  & x2  \\
            (n + 3)  & state & D &  & x3  \\
            (n + 4)  & state & D &  & x4  \\
            (n + 5)  & state & D &  & x5  \\
            (n + 6)  & state & D &  & x6  \\
            (n + 7)  & state & D &  & x7  \\
            (n + 8)  & state & D &  & x8  \\
            (n + 9)  & state & D &  & x9  \\
            (n + 10) & state & D &  & x10 \\
            (n + 11) & state & D &  & x11 \\
            (n + 12) & state & D &  & x12 \\
            (n + 13) & state & D &  & x13 \\
            (n + 14) & state & D &  & x14 \\
            (n + 15) & state & D &  & x15 \\
            (n + 16) & state & D &  & x16 \\
        \end{tabular}
        \hspace{1cm}
        %\end{minipage}%
        %\begin{minipage}{.5\linewidth}
        \begin{tabular}[t]{>{\color{UniRed}}r l >{\color{UniGrey}}l l >{\slshape} l}
            \\
            (n + 17) & state & D   &  & x17    \\
            (n + 18) & state & D   &  & x18    \\
            (n + 19) & state & D   &  & x19    \\
            (n + 20) & state & D   &  & x20    \\
            (n + 21) & state & D   &  & x21    \\
            (n + 22) & state & D   &  & x22    \\
            (n + 23) & state & D   &  & x23    \\
            (n + 24) & state & D   &  & x24    \\
            (n + 25) & state & D   &  & x25    \\
            (n + 26) & state & D   &  & x26    \\
            (n + 27) & state & D   &  & x27    \\
            (n + 28) & state & D   &  & x28    \\
            (n + 29) & state & D   &  & x29    \\
            (n + 30) & state & D   &  & x30    \\
            (n + 31) & state & D   &  & x31    \\
            (n + 32) & state & AS  &  & pc     \\
            (n + 33) & state & Mem &  & memory \\
            \hline
            \hline
        \end{tabular}
        %\end{minipage}
    \end{center}
    \caption[State representation for transforming RISC-V to BTOR2]{State representation for encoding}\label{fig:states}
\end{figure}



\subsection{Initialization}
\subsection{Fetching}
\subsection{Computing Values}
\subsubsection{Opcode}
\subsubsection{funct3 \& funct7}
\subsubsection{Registers}
\subsubsection{Immediate}
\subsection{Command Detection}
\subsection{Next-State Logic}
\subsubsection{$rd$}
\subsubsection{$pc$}
\subsubsection{Memory}
\subsection{Constraints}
\section{Testing for Correctness}
\subsection{State Fuzzer}
\subsection{Automated Logging}

\section{Functional vs Relational Next-State Logic}\label{sec:funcVSrel}