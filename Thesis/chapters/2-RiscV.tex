\chapter{RISC-V}\label{chap:riscv}

As the first foundation for my benchmarks and, consequently, this
thesis, I will discuss RISC-V and its operational principles.

\section{Overview}

RISC-V is an open-source instruction set architecture (ISA) first
published in May 2011 by A. Waterman et al. \cite{first_riscv}. As
indicated by its name, it is based on the RISC design philosophy.
RISC stands for Reduced Instruction Set Computer and its concept is
to only include a small set of elemental and easy to execute
instructions. With this, the decoding and execution of instructions
can be faster compared to CISC (Complex Instruction Set Computer)
design philosophy.

Since 2015, the development of RISC-V has been coordinated by the
RISC-V International Association, a non-profit corporation based in
Switzerland since 2020 \cite{riscvorg}. Its objectives include
providing an \emph{open} ISA that is freely available to all, a
\emph{real} ISA suitable for native hardware implementation, and an
ISA divided into a \emph{small} base integer ISA usable
independently, for example in educational contexts, with optional
standard extensions to support general-purpose software development
\cite[Chapter 1]{riscv-isa}.

Currently, RISC-V comprises four base ISAs: RV32I, RV64I, RV32E, and
RV64E, which can be extended with one or more of the 47 ratified
extension ISAs \cite[Preface]{riscv-isa}.

For the purposes of this work, I focus on a subset of the RV64I ISA.

\section{The RV64I ISA} \label{sec:riscvIsa}
RV64I is not overly complex, but its structure is essential for
understanding the subsequent work presented in this thesis.
Therefore, I will explain all elements relevant to my work.

RV64I features 32 64-bit registers, labeled $x0$–$x31$, where $x0$ is
hardwired to zero across all bits. Registers $x1$–$x31$ are
general-purpose and may be interpreted by various instructions as
collections of booleans, two's complement signed binary integers, or
unsigned integers. Additionally, there is a non-accessible register
called $pc$, which serves as the program counter and holds the
address of the current instruction \cite[Chapters 4.1,
      2.1]{riscv-isa}.

In RV64I, memory addresses are 64 bits in size. As the memory model
is defined to be single-byte addressable, the address space of RV64I
encompasses $2^{64}$ bytes \cite[Chapter 1.4]{riscv-isa}. The format
of the memory is little endian, so the lower bits of a number are
placed at lower addresses.

Like nearly all standard ISAs of RISC-V, RV64I employs a standard
instruction encoding length of 32 bits, or one \emph{word}. Only the
compressed extension named C introduces instructions with a length of
16 bits \cite[Chapter 1.5]{riscv-isa}, but this special case is not
considered here. All RV64I instructions are encoded in one of the six
formats illustrated in \figref{fig:rv64i_formats}. These formats may
consist of
\begin{itemize}
      \item The $opcode$:\\ The opcode is used to differentiate between groups of
            instructions. It also defines the format type of the instruction.
      \item $rd$:\\ This is the destination register.
      \item $funct3$:\\ This is used to differentiate between instructions with
            the same $opcode$.
      \item $rs1$ \& $rs2$:\\ These are the source registers.
      \item $funct7$:\\ This is used for further distinctions between
            instructions if there are more than eight instructions in an opcode
            group and $funct3$ does not suffice.
      \item $imm$:\\ This is an immediate value. In square brackets after $imm$
            is designated a subfield of the immediate which is represented by
            these bits. From these subfields, non-defined lower bits are filled
            with zeros whereas the highest defined bit is sign-extended to fill
            all non-defined higher bits.
\end{itemize}

\begin{figure}[t]
\begin{centering}
    \includegraphics[width = \textwidth]{figures/Encoding-Types/R.png}\\
    \includegraphics[width = \textwidth]{figures/Encoding-Types/I.png}\\
    \includegraphics[width = \textwidth]{figures/Encoding-Types/S.png}\\
    \includegraphics[width = \textwidth]{figures/Encoding-Types/B.png}\\
    \includegraphics[width = \textwidth]{figures/Encoding-Types/U.png}\\
    \includegraphics[width = \textwidth]{figures/Encoding-Types/J.png}
    \caption[RV64I encoding formats]{RV64I encoding formats, used in \cite{riscv-isa}(Chapter 2.3) \todo{Kopie richtig angeben}}
    \label{fig:rv64i_formats}
\end{centering}
\end{figure}


The design of these formats results in the following features:
\begin{itemize}
      \item Due to RISC-V's little-endian nature, the $opcode$, which encodes the
            general instruction, is always read first. Further specification of
            the instruction via $funct3$ and $funct7$ is consistently located at
            the same positions.
      \item If utilized by the instruction, $rd$, $rs1$, and $rs2$ are also
            always found in the same locations, simplifying decoding.
      \item The highest bit of $imm$ is always bit 31, making it straightforward
            to sign-extend the immediate value.
\end{itemize}

The instructions relevant to my work are listed in
\tabref{tab:rv64i-instructions} I have divided the instructions in
\tabref{tab:rv64i-instructions} into nine groups based on their
operations.

\texttt{LUI} and \texttt{AUIPC} move a high immediate into $rd$. In
the case of \texttt{AUIPC}, the $pc$ is added to this value.
\texttt{JAL} and \texttt{JALR} instructions are unconditional jumps,
where for \texttt{JAL} $imm$ is added to $pc$ and for \texttt{JALR}
$imm$ is added to $rs1$ and set as $pc$. Both link to the next
instruction (current $pc + 4$) in $rd$.

$branch$ instructions are conditional jumps. $rs1$ is compared to
$rs2$ and if the comparison holds, $imm$ is added to $pc$. The
comparisons are $=$ for \texttt{BEQ}, $\neq$ for \texttt{BNE}, $<$
for \texttt{BLT}, and $\ge$ for \texttt{BGE}. In these instructions,
the values in $rs1$ and $rs2$ are handled as two's complement
integers. The suffix \texttt{$^*$U} in an instruction generally
designates an unsigned operation. In this case, the values in $rs1$
and $rs2$ are handled as unsigned integers. Apart from this, they
work as their counterpart without the suffix.

$load$ instructions load values from memory at address $(rs1+imm)$
into $rd$, either at \texttt{B}yte, \texttt{H}alfword, \texttt{W}ord,
or \texttt{D}oubleword length. By default, the value is
sign-extended, and the suffix \texttt{$^*$U} designates the loading
of a non-sign-extended value. Conversely, $store$ instructions write
values from $rs2$ at the address $(rs1+imm)$ to memory. Here also the
distinction between the different lengths is made, and the lowest
byte, halfword, word, or the whole doubleword is stored at the
address. \begin{tabular}{ll}

\end{tabular}

All further instructions can be seen as generic operations,
differentiated by their suffixes. To simplify the explanation
process, all operations without any suffix and their behavior are
listed in \tabref{tab:operations}. This is almost exactly the group
with opcode $op$, except the \texttt{SLTU} instruction, which is not
suffix-free. However, as with all other instructions with the
unsigned suffix, it behaves as its signed counterpart except for
handling both $rs1$ and $rs2$ as unsigned integers.

\begin{table}
    \centering
    \begin{tabular}{>{\ttfamily}r|l}
        \hline
        Instr & Behavior                                               \\
        \hline
        ADD   & $rd := rs1 + rs2$                                      \\
        SUB   & $rd := rs1 - rs2$                                      \\
        SLT   & $rd := 1$ if $rs1 < rs2$ else $rd := 0$                \\
        XOR   & $rd := rs1 \oplus rs2$, bitwise                        \\
        OR    & $rd := rs1 \vee rs2$, bitwise                          \\
        AND   & $rd := rs1 \wedge rs2$, bitwise                        \\
        SLL   & $rd := rs1$ shifted left by $rs2$, new bits are zeros  \\
        SRL   & $rd := rs1$ shifted right by $rs2$, new bits are zeros \\
        SRA   & $rd := rs1$ shifted right by $rs2$, sign extend        \\
    \end{tabular}
    \caption[Behavior of RV64I Operations]{All Suffix-free
        Operations in RV64I and their Behavior. All Values are handled either bitwise or as signed twos-complement Integers}\label{tab:operations}
\end{table}

These operations can be extended by the \texttt{$^*$I} suffix, which
is designated by the opcode $op-imm$. This replaces $rs2$ with $imm$
in the behavior. Again, \texttt{SLTI} can be extended to an unsigned
version \texttt{SLTIU}, which behaves as expected. A \texttt{SUBI}
instruction does not exist as it is redundant; its behavior can be
achieved by using \texttt{ADDI} with a negative immediate.

Additionally, the operations \texttt{ADD}, \texttt{SUB},
\texttt{SLL}, \texttt{SRL}, and \texttt{SRA} can be extended with the
\texttt{$^*$W} suffix. This forms the group with the opcode $op-32$.
In contrast to the base instructions, these new ones behave as if the
registers are only 32 bits. The result is placed in the low 32 bits
of $rd$ and sign-extended to the full 64 bits. Overflows are ignored.

The last group is the combination of both suffixes \texttt{$^*$IW}
with the opcode $op-imm-32$. The behavior differs from the base
instructions, as expected, by a replacement of $rs2$ with $imm$ and
only operating on 32 bits. Again, a \texttt{SUBIW} instruction is
redundant as a negative immediate with \texttt{ADDIW} achieves the
same result.

Compared to the full RV64I ISA, I have omitted the \texttt{FENCE},
\texttt{ECALL}, and \texttt{EBREAK} instructions, as without I/O
interaction or an environment such as an OS or a debugger, these are
not required.

For each of the instructions in \tabref{tab:operations}, I also
included the format in which each instruction is encoded. Most should
be not surprising as they fit the description of the instructions.
Only \texttt{SLLI, SRLI, SRAI} with I$^*$ and \texttt{SLLIW, SRLIW,
      SRAIW} with I$^{**}$ should need clarification. Both are essentialy
the I format but with extra constraints. For I$^*$ the highest
realistic shift amount for 64 bit registers is also 64. So the bits
      [11,9:6] of $imm$ have to be 0. The bit [10] gets a special role as
it is used to differentiate between the two types of right shift.
With I$^{**}$, with the word suffix, the maximum shift amount is only
32, so the bit [5] of $imm$ must also be 0.

\section{Simulation of RISC-V}\label{sec:simulation}
To run RISC-V code, the obvious way would be to run it on a RISC-V
processor. As this is not practical in my case because I do not have
one, I have to simulate the execution of RISC-V. For this, I will use
a RISC-V-simulator of this described subset of RV64I in C that I have
written for my bachelors project \cite{repoSim}. It will be used to
test the BTOR2 model I will present in \chapref{chap:riscv_to_btor2}.
Alternatively qemu \cite{qemu} or QtRVSim \cite{qtrvsim} could be
used, but QtRVSim has more detail than needed and qemu can not be fed
with an inital state but only a full program. So I used my own, which
takes the state of a RISC-V processor as an input as described in
\secref{sub:statefile}.

First, I explain the structure to represent a simple RISC-V processor
I implemented.

\subsection{Representing the State of a RISC-V Processor}
The state requires a representation for all registers. $pc$ is
defined as a 64-bit integer, and the other 32 registers are
implemented as an array, allowing each register to be referenced by
its number. Additionally, I implemented an array of flags, one for
each register, to differentiate between initialized and
non-initialized registers. The memory is built from single memory
cells, each holding an address and its byte of content. These are
accumulated in a hash table called \enquote{memorytable}, hashing on
the address. If adding a new cell causes a collision, it is appended
to the cells already in the bucket, forming a linked list. These
structures are shown in \figref{fig:structs}.

\begin{figure}
    \centering
    \begin{minipage}[b]{0.48 \linewidth}
        \inputminted[bgcolor=LightGray]{c}{figures/2-RiscV/structs.c}
    \end{minipage}\hspace{0.01 \linewidth}
    \begin{minipage}[b]{0.48 \linewidth}
        \inputminted[bgcolor=LightGray]{c}{figures/2-RiscV/structs2.c}
    \end{minipage}\vspace{0.01 \linewidth}
    \begin{minipage}[t]{0.48 \linewidth}
        \inputminted[bgcolor=LightGray]{c}{figures/2-RiscV/structs3.c}
    \end{minipage}
    \caption[State representation of RISC-V in c]{State representation of a RISC-V processor in the simulation \cite{repoSim}}\label{fig:structs}
\end{figure}

\subsection{Running an Instruction}\label{sec:runInstr}
After fetching the current instruction from the hash table, it must
be decoded. The easiest way to decode the operation corresponding to
the current instruction is a decision tree like in \figref{fig:tree}.
First, I mask out the opcode and match it over all implemented
opcodes. From there, either this is an endpoint and the instruction
is identified, or $funct3$ must be masked and matched. A final
differentiation over $funct7$ might be needed, but after this every
leaf in the tree coincides with an instruction. Also, with knowing
the opcode of the current instruction, I know the instruction format
(\figref{fig:rv64i_formats}). This means that I can now also extract
relevant register numbers and, if it exists, the immediate. The best
way to get these values is to apply a mask and shift this result to
its correct place. For the immediate, as it possibly is divided into
multiple fields, it might be needed to add multiple partial
immediates together. At this point all information in the instruction
is decoded and the current state can be modified according to the
operation corresponding to the leaf reached after going through the
tree.


\begin{figure}
    \centering
    \begin{forest}
        for descendants={grow=east, anchor=west, child anchor=west, s sep=0},
        grow=east, anchor=west, s sep=0,
        [opcode:
        [\ttfamily 3, draw
        [$load$, tier=opcode
        [funct3:, tier=f3
        [\ttfamily 0, draw
            [\ttfamily LB, tier=instr]
        ]
        [\ttfamily 1, draw
            [\ttfamily LH, tier=instr]
        ]
        [\ttfamily 2, draw
            [\ttfamily LW, tier=instr]
        ]
        [\ttfamily 3, draw
            [\ttfamily LD, tier=instr]
        ]
        [\ttfamily 4, draw
            [\ttfamily LBU, tier=instr]
        ]
        [\ttfamily 5, draw
            [\ttfamily LHU, tier=instr]
        ]
        [\ttfamily 6, draw
            [\ttfamily LWU, tier=instr]
        ]
        ]]
        ]
        [\ttfamily 19, draw
        [$opImm$, tier=opcode
        [funct3:, tier=f3
        [\ttfamily 0, draw
            [\ttfamily ADDI, tier=instr]
        ]
        [\ttfamily 1, draw
            [\ttfamily SLLI, tier=instr]
        ]
        [\ttfamily 2, draw
            [\ttfamily SLTI, tier=instr]
        ]
        [\ttfamily 3, draw
            [\ttfamily SLTIU, tier=instr]
        ]
        [\ttfamily 4, draw
            [\ttfamily XORI, tier=instr]
        ]
        [\ttfamily 5, draw
        [funct7 (only bit 30):, tier=f7
        [\ttfamily 0, draw [\ttfamily SRLI]]
        [\ttfamily 1, draw [\ttfamily SRAI]]]
        ]
        [\ttfamily 6, draw
            [\ttfamily ORI, tier=instr]
        ]
        [\ttfamily 7, draw
            [\ttfamily ANDI, tier=instr]]
        ]]
        ]
        [\ttfamily 23, draw
        [$auipc$, tier=opcode
        [\ttfamily AUIPC, tier=instr2]
        ]
        ]
        [\ttfamily 27, draw
        [$opImm32$, tier=opcode
        [funct3:, tier=f3
        [\ttfamily 0, draw
            [\ttfamily ADDIW, tier=instr]
        ]
        [\ttfamily 1, draw
            [\ttfamily SLLIW, tier=instr]
        ]
        [\ttfamily 5, draw
        [funct7 (only bit 30):, tier=f7
        [\ttfamily 0, draw [\ttfamily SRLIW]]
        [\ttfamily 1, draw [\ttfamily SRAIW]]]
        ]
        ]]]
        [\ttfamily 35, draw
        [$store$, tier=opcode
        [funct3:, tier=f3
        [\ttfamily 0, draw
            [\ttfamily SB, tier=instr]
        ]
        [\ttfamily 1, draw
            [\ttfamily SH, tier=instr]
        ]
        [\ttfamily 2, draw
            [\ttfamily SW, tier=instr]
        ]
        [\ttfamily 3, draw
            [\ttfamily SD, tier=instr]
        ]]]
        ]
        [\ttfamily 51, draw
        [$op$, tier=opcode
        [funct3:, tier=f3
        [\ttfamily 0, draw
        [funct7 (only bit 30):, tier=f7
        [\ttfamily 0, draw [\ttfamily ADD]]
        [\ttfamily 1, draw [\ttfamily SUB]]
        ]
        ]
        [\ttfamily 1, draw
            [\ttfamily SLL, tier=instr]
        ]
        [\ttfamily 2, draw
            [\ttfamily SLT, tier=instr]
        ]
        [\ttfamily 3, draw
            [\ttfamily SLTU, tier=instr]
        ]
        [\ttfamily 4, draw
            [\ttfamily XOR, tier=instr]
        ]
        [\ttfamily 5, draw
        [funct7 (only bit 30):, tier=f7 [\ttfamily 0, draw [\ttfamily SRL]] [\ttfamily 1,
        draw [\ttfamily SRA]]] ] [\ttfamily 6, draw [\ttfamily OR,
                tier=instr] ] [\ttfamily 7, draw [\ttfamily AND, tier=instr]]]] ]
        [\ttfamily 55, draw [$lui$, tier=opcode [\ttfamily LUI, tier=instr2] ]
        ] [\ttfamily 59, draw [$op32$, tier=opcode [funct3:, tier=f3
        [\ttfamily 0, draw [funct7 (only bit 30):, tier=f7
        [\ttfamily 0, draw [\ttfamily ADDW]]
        [\ttfamily 1, draw [\ttfamily SUBW]]
        ] ] [\ttfamily 1, draw
            [\ttfamily SLLW, tier=instr] ] [\ttfamily 5, draw [funct7 (only bit 30):,
        tier=f7
        [\ttfamily 0, draw [\ttfamily SRLW]]
        [\ttfamily 1, draw [\ttfamily SRAW]]] ]]] ] [\ttfamily 99, draw [$branch$, tier=opcode [funct3:,
        tier=f3 [\ttfamily 0, draw [\ttfamily BEQ, tier=instr] ] [\ttfamily
        1, draw [\ttfamily BNE, tier=instr] ] [\ttfamily 4, draw [\ttfamily
                BLT, tier=instr] ] [\ttfamily 5, draw [\ttfamily BGE, tier=instr] ]
        [\ttfamily 6, draw [\ttfamily BLTU, tier=instr] ] [\ttfamily 7, draw
            [\ttfamily BGEU, tier=instr]] ]] ] [\ttfamily 103, draw [$jalr$,
        tier=opcode [\ttfamily JALR, tier=instr2] ] ] [\ttfamily 111, draw
        [$jal$, tier=opcode [\ttfamily JAL, tier=instr2] ] ] ]
    \end{forest}
    \caption[Decision Tree to find the right Operation]{Decision Tree to find the right Operation based on the current Instruction}\label{fig:tree}
\end{figure}

\subsection{Saving the State of a RISC-V
      Processor}\label{sub:statefile}

To preserve the current state of a RISC-V processor, both the
registers and memory must be stored. For this purpose, I have devised
the format shown in \figref{statefileform}. The RISC-V simulation
uses this format as input and output. The minimal file consists only
of the two designators \enquote{REGISTERS:} and \enquote{MEMORY:} and
one empty line between them. Under \enquote{REGISTERS:}, all
registers can be listed with their corresponding value. Of course, x0
cannot be different from 0. I included the option to reference it
nonetheless to have the complete state included. Under
\enquote{MEMORY:}, after giving an address, the memory can be filled
with 1-, 2-, 4-, or 8-byte sized memory content. The given address is
the starting address of the content. As RISC-V is little-endian, the
rightmost byte is placed at the starting address. From there the next
byte on the left is placed at the next higher address. Additionally,
comments can be added after a \#. The total length of the line should
not extend over 80 characters.

\begin{figure}
    \begin{tabular}{ll}
        1 & REGISTERS:                                                                    \\
        2 & PC: \textit{current pc in hex}                                                \\
        3 & x\textit{(0 - 31)}: \textit{value of register in hex}                         \\
        4 &                                                                               \\
        5 & MEMORY:                                                                       \\
        6 & \textit{(address in hex)}: \textit{byte, halfword, word or doubleword in hex}
    \end{tabular}
    \caption[Construction of .state files]{Construction of .state files}
    \label{statefileform}
\end{figure}