\chapter{RISC-V}\label{chap:riscv}

\section{Overview}

RISC-V is an open source instruction set architecture first published in May
2011 by A. Waterman et al. \cite{first_riscv}. As contained in the name, it is
based on the RISC design philosophy. \todo{Explain RISK (compare wiki)} Since
2015 the development of RISC-V is coordinated by the RISC-V International
Association, a non-profit corporation based in Switzerland since 2020
\cite{riscvorg}. Its goals are among others an \emph{open} ISA that is freely
available to all, a \emph{real} ISA suitable for native hardware implementation
and an ISA separated into a \emph{small} base integer ISA usable by itself e.g.
for educational purpose and optional standard extensions to support general
purpose software development \cite{riscv-isa}(Chapter 1).

It currently contains four base ISAs, namely RV32I, RV64I, RV32E and RV64E,
which may be extended with one or more of the 47 ratified extension ISAs
\cite{riscv-isa} (Preface).

\extend{Hier brauchts vlt noch was} \todo{little endian erwähnen?}

For my work, I will focus on a subset of the RV64I ISA.

\section{The RV64I ISA}
RV64I is not complex, but its structure is relevant to understand my later
work. So I explain all elements relevant for my thesis.

RV64I has 32 64bit registers called $x0$-$x31$, where $x0$ is hardwired to 0 on
all bits. Registers $x1$-$x31$ are general purpose and are interpreted by
various instructions as a collection of booleans, two's complement signed
binary integers or unsigned integers. Additionally, there is a register called
$pc$ acting as program counter and holding the address of the current
instruction \cite{riscv-isa}(Chapters 4.1, 2.1).

In RV64I, a memory address has the size of 64bit. As the memory model is
defined to be single byte addressable, the address space of RV64I is $2^{64}$
bytes \cite{riscv-isa}(Chapter 1.4).

Like almost all the standard ISAs of RISC-V, RV64I has a standard encoding
length of 32bit or 1 \emph{word}. Only the compressed extension C adds
instructions with a length of 16bit \cite{riscv-isa}(Chapter 1.5), but it is
irrelevant for us. All instructions of RV64I are encoded in one of the six
formats shown in \figref{fig:rv64i_formats}.

\begin{figure}[t]
\begin{centering}
    \includegraphics[width = \textwidth]{figures/Encoding-Types/R.png}\\
    \includegraphics[width = \textwidth]{figures/Encoding-Types/I.png}\\
    \includegraphics[width = \textwidth]{figures/Encoding-Types/S.png}\\
    \includegraphics[width = \textwidth]{figures/Encoding-Types/B.png}\\
    \includegraphics[width = \textwidth]{figures/Encoding-Types/U.png}\\
    \includegraphics[width = \textwidth]{figures/Encoding-Types/J.png}
    \caption[RV64I encoding formats]{RV64I encoding formats, used in \cite{riscv-isa}(Chapter 2.3) \todo{Kopie richtig angeben}}
    \label{fig:rv64i_formats}
\end{centering}
\end{figure}


The design of these format results in the following features:
\begin{itemize}
    \item As of RISC-Vs little endianess, the $opcode$, which encodes the general
          instruction, is always read first. Also, further specification of the
          instruction by $funct3$ and $funct7$ is found always at the same location.
    \item If used by the instruction, the destination register $rd$ and the source
          registers $rs1$ \& $rs2$ are always at the same place. This simplifies
          decoding.
    \item The highest bit of the immediate value $imm$ is always bit 31. This makes
          finding the sign of a signed immediate value trivial.
\end{itemize}
Note that each immediate subfield is labeled with the bit position in the immediate value.
Immediate values are always sign extended to 31bit and in case of U-, B- and J-type the missing lower bits are filled with zeros.

The instructions relevant to my work are listed in
\tabref{tab:rv64i-instructions} \begin{tabular}{ll}

\end{tabular}

\section{Simulation of RISC-V}

\subsection{Saving the State of a RISC-V Processor}\label{statefile}