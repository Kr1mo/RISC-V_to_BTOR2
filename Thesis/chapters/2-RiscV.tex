\chapter{RISC-V}\label{chap:riscv}

As the first foundation for my benchmarks and, consequently, this
thesis, I will discuss RISC-V and its operational principles.

\section{Overview}

RISC-V is an open-source instruction set architecture first published
in May 2011 by A. Waterman et al. \cite{first_riscv}. As indicated by
its name, it is based on the RISC design philosophy. \todo{Explain
      RISC (compare wiki)} Since 2015, the development of RISC-V has been
coordinated by the RISC-V International Association, a non-profit
corporation based in Switzerland since 2020 \cite{riscvorg}. Its
objectives include providing an \emph{open} ISA that is freely
available to all, a \emph{real} ISA suitable for native hardware
implementation, and an ISA divided into a \emph{small} base integer
ISA usable independently, for example in educational contexts, with
optional standard extensions to support general-purpose software
development \cite[Chapter 1]{riscv-isa}.

Currently, RISC-V comprises four base ISAs: RV32I, RV64I, RV32E, and
RV64E, which can be extended with one or more of the 47 ratified
extension ISAs \cite[Preface]{riscv-isa}.

\extend{Additional content may be required here} \todo{Mention little
      endian?}

For the purposes of this work, I will focus on a subset of the RV64I
ISA.

\section{The RV64I ISA} \label{sec:riscvIsa}
RV64I is not overly complex, but its structure is essential for
understanding the subsequent work presented in this thesis.
Therefore, I will explain all elements relevant to my work.

RV64I features 32 64-bit registers, labeled $x0$–$x31$, where $x0$ is
hardwired to zero across all bits. Registers $x1$–$x31$ are
general-purpose and may be interpreted by various instructions as
collections of booleans, two's complement signed binary integers, or
unsigned integers. Additionally, there is a non-accessible register
called $pc$, which serves as the program counter and holds the
address of the current instruction \cite[Chapters 4.1,
      2.1]{riscv-isa}.

In RV64I, memory addresses are 64 bits in size. As the memory model
is defined to be single-byte addressable, the address space of RV64I
encompasses $2^{64}$ bytes \cite[Chapter 1.4]{riscv-isa}.

Like nearly all standard ISAs of RISC-V, RV64I employs a standard
instruction encoding length of 32 bits, or one \emph{word}. Only the
compressed extension named C introduces instructions with a length of
16 bits \cite[Chapter 1.5]{riscv-isa}, but we will not encounter this
special case. All RV64I instructions are encoded in one of the six
formats illustrated in \figref{fig:rv64i_formats}. These formats may
consist of
\begin{itemize}
      \item The $opcode$:\\ The opcode is used to differ between groups of
            instructions. It also defines the format type of the instruction.
      \item $rd$:\\ This is the destination register.
      \item $funct3$:\\This is used to differ between instructions with the same
            $opcode$.
      \item $rs1$ \& $rs2$:\\ These are the source registers.
      \item $funct7$\\:This is used for further distinctions between instructions
            if there are more than 8 instructions in an opcode group and $funct3$
            does not suffice.
      \item $imm$:\\This is an immediate value. In square brackets after $imm$ is
            designated a subfield of the immediate which is represented by these
            bits. From these subfields, non-defined lower bits are filled with
            zeros whereas the highest defined bit is sign extended to fill all
            non-defined higher bits.
\end{itemize}

\begin{figure}[t]
\begin{centering}
    \includegraphics[width = \textwidth]{figures/Encoding-Types/R.png}\\
    \includegraphics[width = \textwidth]{figures/Encoding-Types/I.png}\\
    \includegraphics[width = \textwidth]{figures/Encoding-Types/S.png}\\
    \includegraphics[width = \textwidth]{figures/Encoding-Types/B.png}\\
    \includegraphics[width = \textwidth]{figures/Encoding-Types/U.png}\\
    \includegraphics[width = \textwidth]{figures/Encoding-Types/J.png}
    \caption[RV64I encoding formats]{RV64I encoding formats, used in \cite{riscv-isa}(Chapter 2.3) \todo{Kopie richtig angeben}}
    \label{fig:rv64i_formats}
\end{centering}
\end{figure}


The design of these formats results in the following features:
\begin{itemize}
      \item Due to RISC-V's little-endian nature, the $opcode$, which encodes the
            general instruction, is always read first. Further specification of
            the instruction via $funct3$ and $funct7$ is consistently located at
            the same positions.
      \item If utilized by the instruction, $rd$, $rs1$ and $rs2$ are also always
            found in the same locations, simplifying decoding.
      \item The highest bit of $imm$ is always bit 31, making it straightforward
            to sign-extend the immediate value.
\end{itemize}

The instructions relevant to my work are listed in
\tabref{tab:rv64i-instructions}
\begin{tabular}{ll}

\end{tabular}

I have divided the instructions in \tabref{tab:rv64i-instructions}
into nine groups based on their operations.

\texttt{LUI} and \texttt{AUIPC} move a high immediate into $rd$. In
case of \texttt{AUIPC}, the $pc$ is added onto this.

\texttt{JAL} and \texttt{JALR} instructions are unconditional jumps,
where on \texttt{JAL} $imm$ is added onto $pc$ and on \texttt{JALR}
$imm$ is added onto $rs1$ and set as $pc$. Both link to the next
instruction (current $pc + 4$) in $rd$.

$branch$ instructions are conditional jumps. $rs1$ is compared to
$rs2$ and if the comparison holds, $imm$ is added onto $pc$. The
comparison are $=$ for \texttt{BEQ}, $\neq$ for \texttt{BNE}, $<$ for
\texttt{BLT} and $\ge$ for \texttt{BGE}. In these instructions, the
values in $rs1$ and $rs2$ are handled as twos complement integers.
The suffix \texttt{$^*$U} in an instruction generally designates an
unsigned operation. In this case the values in $rs1$ and $rs2$ are
handled as unsigned integers. Apart from this, they work as their
counterpart without the suffix.

$load$ instructions load values from memory at address $(rs1+imm)$
into $rd$, either at \texttt{B}yte, \texttt{H}alfword, \texttt{W}ord,
or \texttt{D}oubleword length. Again the standard is a sign-extended
value and the suffix \texttt{$^*$U} designates the loading of a non
sign extended value.

Conversely, $store$ instructions write values from $rs2$ at the
address $(rs1+imm)$ to memory. Here also the distinction between the
different lengths is made and the lowest byte, halfword, word or the
whole doubleword is stored at the address.

All further instructions can be seen as generic operations,
differentiated by their suffixes. To simplify the explanation
process, all operations without any suffix and their behavior are
listed in \tabref{tab:operations}. This is almost exactly the group
with opcode $op$, except the \texttt{SLTU} instruction, which is not
suffix free. But as all other instructions with the unsigned suffix
it behaves as its signed counterpart except from handling both $rs1$
and $rs2$ as unsigned integers.

\begin{table}
    \centering
    \begin{tabular}{>{\ttfamily}r|l}
        \hline
        Instr & Behavior                                               \\
        \hline
        ADD   & $rd := rs1 + rs2$                                      \\
        SUB   & $rd := rs1 - rs2$                                      \\
        SLT   & $rd := 1$ if $rs1 < rs2$ else $rd := 0$                \\
        XOR   & $rd := rs1 \oplus rs2$, bitwise                        \\
        OR    & $rd := rs1 \vee rs2$, bitwise                          \\
        AND   & $rd := rs1 \wedge rs2$, bitwise                        \\
        SLL   & $rd := rs1$ shifted left by $rs2$, new bits are zeros  \\
        SRL   & $rd := rs1$ shifted right by $rs2$, new bits are zeros \\
        SRA   & $rd := rs1$ shifted right by $rs2$, sign extend        \\
    \end{tabular}
    \caption[Behavior of RV64I Operations]{All Suffix-free
        Operations in RV64I and their Behavior. All Values are handled either bitwise or as signed twos-complement Integers}\label{tab:operations}
\end{table}

These operations can be extended by the \texttt{$^*$I} suffix which
is designated by the opcode $op-imm$. This exchanges $rs2$ with $imm$
in the behavior. Again, \texttt{SLTI} can be extended to an unsigned
version \texttt{SLTIU}, which behaves as expected. A \texttt{SUBI}
instruction does not exist as it is redundant. Its behavior can be
reached by using \texttt{ADDI} with a negative immediate.

Also, the operations \texttt{ADD}, \texttt{SUB}, \texttt{SLL},
\texttt{SRL} and \texttt{SRA} can be extended with the \texttt{$^*$W}
suffix. This forms the group with the opcode $op-32$. In contrast to
the base instructions these new ones behave as if the registers are
only 32bit. The result is placed in the low 32 bits of $rd$ and sign
extended to full 64bit. Overflows are ignored.

The last group left is the combination of both suffixes
\texttt{$^*$IW} with the opcode $op-imm-32$. The behavior differs
from the base instructions, as expected, by a replacement of $rs2$
with $imm$ and only operating on 32bit. Again, a \texttt{SUBIW}
instruction is redundant as a negative immediate with \texttt{ADDIW}
archives the same.

Compared to the full RV64I ISA, I left out \texttt{FENCE},
\texttt{ECALL} and \texttt{EBREAK} instructions as without I/O
interaction or an environment like an OS or a debugger, these are not
needed.

\section{Simulation of RISC-V}\label{sec:simulation}
In a previous project, I have written a simulation of this subset of
RV64I in C \cite{repoSim}. As I will use it to later on test my BTOR2
model, I explain the inner working of the program here.

First I implemented a structure to represent a simple RISC-V
processor

\subsection{Representing the State of a RISC-V Processor}
Of course the state needs a representation for all registers. $pc$ is
defined as a 64bit integer, the other 32 registers are implemented as
an array, as so each register can be referenced by its number. Also,
I implemented an array of flags, one for each register, to be able to
differ between initialized and non initialized registers. The memory
is build from single memory cells holding each an address and its
byte of content. These are cumulated in a hash table
\enquote{memorytable}, hashing on the address. If adding a new cell
causes a collision, it is appended to the cells already in the bucket
forming a linked list. These structs are shown in
\figref{fig:structs}.

\begin{figure}
    \centering
    \begin{minipage}[b]{0.48 \linewidth}
        \inputminted[bgcolor=LightGray]{c}{figures/2-RiscV/structs.c}
    \end{minipage}\hspace{0.01 \linewidth}
    \begin{minipage}[b]{0.48 \linewidth}
        \inputminted[bgcolor=LightGray]{c}{figures/2-RiscV/structs2.c}
    \end{minipage}\vspace{0.01 \linewidth}
    \begin{minipage}[t]{0.48 \linewidth}
        \inputminted[bgcolor=LightGray]{c}{figures/2-RiscV/structs3.c}
    \end{minipage}
    \caption[State representation of RISC-V in c]{State representation of a RISC-V processor in the simulation \cite{repoSim}}\label{fig:structs}
\end{figure}

\subsection{Instruction detection}
After fetching the current instruction from the hash table, it must
be analyzed to detect the current instruction.

\subsection{Instruction execution}

\subsection{Saving the State of a RISC-V
      Processor}\label{sub:statefile}

To preserve the current state of a RISC-V processor, both the
registers and memory must be stored. For this purpose, I have devised
the format shown in \figref{statefileform}. The minimal file consists
only of the two designators \enquote{REGISTERS:} and
\enquote{MEMORY:}, the current $pc$, and one empty line.

\begin{figure}
    \begin{tabular}{ll}
        1 & REGISTERS:                                                                    \\
        2 & PC: \textit{current pc in hex}                                                \\
        3 & x\textit{(0 - 31)}: \textit{value of register in hex}                         \\
        4 &                                                                               \\
        5 & MEMORY:                                                                       \\
        6 & \textit{(address in hex)}: \textit{byte, halfword, word or doubleword in hex}
    \end{tabular}
    \caption[Construction of .state files]{Construction of .state files}
    \label{statefileform}
\end{figure}