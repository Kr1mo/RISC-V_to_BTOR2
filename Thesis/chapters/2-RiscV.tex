\chapter{RISC-V}\label{chap:riscv}

As the first foundation for my benchmarks and, consequently, this thesis, I
will discuss RISC-V and its operational principles.

\section{Overview}

RISC-V is an open-source instruction set architecture first published in May
2011 by A. Waterman et al. \cite{first_riscv}. As indicated by its name, it is
based on the RISC design philosophy. \todo{Explain RISC (compare wiki)} Since
2015, the development of RISC-V has been coordinated by the RISC-V
International Association, a non-profit corporation based in Switzerland since
2020 \cite{riscvorg}. Its objectives include providing an \emph{open} ISA that
is freely available to all, a \emph{real} ISA suitable for native hardware
implementation, and an ISA divided into a \emph{small} base integer ISA usable
independently, for example in educational contexts, with optional standard
extensions to support general-purpose software development \cite[Chapter
      1]{riscv-isa}.

Currently, RISC-V comprises four base ISAs: RV32I, RV64I, RV32E, and RV64E,
which can be extended with one or more of the 47 ratified extension ISAs
\cite[Preface]{riscv-isa}.

\extend{Additional content may be required here} \todo{Mention little endian?}

For the purposes of this work, I will focus on a subset of the RV64I ISA.

\section{The RV64I ISA} \label{sec:riscvIsa}
RV64I is not overly complex, but its structure is essential for understanding
the subsequent work presented in this thesis. Therefore, I will explain all
elements relevant to my research.

RV64I features 32 64-bit registers, labeled $x0$–$x31$, where $x0$ is hardwired
to zero across all bits. Registers $x1$–$x31$ are general-purpose and may be
interpreted by various instructions as collections of booleans, two's
complement signed binary integers, or unsigned integers. Additionally, there is
a register called $pc$, which serves as the program counter and holds the
address of the current instruction \cite[Chapters 4.1, 2.1]{riscv-isa}.

In RV64I, memory addresses are 64 bits in size. As the memory model is defined
to be single-byte addressable, the address space of RV64I encompasses $2^{64}$
bytes \cite[Chapter 1.4]{riscv-isa}.

Like nearly all standard ISAs of RISC-V, RV64I employs a standard instruction
encoding length of 32 bits, or one \emph{word}. Only the compressed extension C
introduces instructions with a length of 16 bits \cite[Chapter 1.5]{riscv-isa},
which is not relevant for this discussion. All RV64I instructions are encoded
in one of the six formats illustrated in \figref{fig:rv64i_formats}.

\begin{figure}[t]
\begin{centering}
    \includegraphics[width = \textwidth]{figures/Encoding-Types/R.png}\\
    \includegraphics[width = \textwidth]{figures/Encoding-Types/I.png}\\
    \includegraphics[width = \textwidth]{figures/Encoding-Types/S.png}\\
    \includegraphics[width = \textwidth]{figures/Encoding-Types/B.png}\\
    \includegraphics[width = \textwidth]{figures/Encoding-Types/U.png}\\
    \includegraphics[width = \textwidth]{figures/Encoding-Types/J.png}
    \caption[RV64I encoding formats]{RV64I encoding formats, used in \cite{riscv-isa}(Chapter 2.3) \todo{Kopie richtig angeben}}
    \label{fig:rv64i_formats}
\end{centering}
\end{figure}


The design of these formats results in the following features:
\begin{itemize}
      \item Due to RISC-V's little-endian nature, the $opcode$, which encodes the general
            instruction, is always read first. Further specification of the instruction via
            $funct3$ and $funct7$ is consistently located at the same positions.
      \item If utilized by the instruction, the destination register $rd$ and the source
            registers $rs1$ and $rs2$ are always found in the same locations, simplifying
            decoding.
      \item The highest bit of the immediate value $imm$ is always bit 31, making it
            straightforward to sign-extend the immediate value.
\end{itemize}
Note that each immediate subfield is labeled with its bit position within the immediate value.
Immediate values are always sign-extended to 31 bits, and in the case of U-, B-, and J-type formats, the missing lower bits are filled with zeros.

The instructions relevant to my work are listed in
\tabref{tab:rv64i-instructions} \begin{tabular}{ll}

\end{tabular}

I have divided the instructions in \tabref{tab:rv64i-instructions} into nine
groups based on their operations. \texttt{LUI} and \texttt{AUIPC} move a high
immediate into $rd$; \texttt{JA*} instructions are unconditional jumps, and
\texttt{B*} instructions are conditional jumps. \texttt{L*} instructions load
sign-extended values from memory, either as \texttt{B}yte, \texttt{H}alfword,
\texttt{W}ord, or \texttt{D}oubleword lengths. Conversely, \texttt{S*}
instructions write values of the specified length to memory. \todo{arithmetic}
Note that the suffix \texttt{U} denotes operations where values are processed
as unsigned.

I left out \texttt{FENCE}, \texttt{ECALL} and \texttt{EBREAK} instructions as
without I/O interaction or an environment like an OS or a debugger, these are
not needed.

\section{Simulation of RISC-V}\label{sec:simulation}
\todo{This may be better placed in Chapter 4, but the state file is relevant here.}
\cite{repoSim}

\subsection{Representing the State of a RISC-V Processor}
\subsection{Instruction detection}
\subsection{Instruction execution}

\subsection{Saving the State of a RISC-V Processor}\label{sub:statefile}
To preserve the current state of a RISC-V processor, both the registers and
memory must be stored. For this purpose, I have devised the format shown in
\figref{statefileform}. The minimal file consists only of the two designators
\dq REGISTERS:\dq\ and \dq MEMORY:\dq, the current $pc$, and one empty line.
\begin{figure}
    \begin{tabular}{ll}
        1 & REGISTERS:                                                                    \\
        2 & PC: \textit{current pc in hex}                                                \\
        3 & x\textit{(0 - 31)}: \textit{value of register in hex}                         \\
        4 &                                                                               \\
        5 & MEMORY:                                                                       \\
        6 & \textit{(address in hex)}: \textit{byte, halfword, word or doubleword in hex}
    \end{tabular}
    \caption[Construction of .state files]{Construction of .state files}
    \label{statefileform}
\end{figure}